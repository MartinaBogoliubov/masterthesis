%
%
\chapter{Properties of the Kubo Relaxation Function}
\label{app:properties of the Kubo relaxation function}
%
%
In section \ref{subsec: Kubo relaxation function} the Kubo relaxation function 
%
\begin{align}
	\Phi_{\mt{AB}}(t) = \frac{i}{\hbar} \lim\limits_{s \to 0} \int\limits_{t}^{\infty} \dd{\tau} \expval{\comm{\mt{A}_{\mt{I}}(\tau)}{\mt{B}_{\mt{I}}(0)}}_{0} e^{-s\tau}.
	\label{appeq: Kubo relaxation function}
\end{align}
%
and the three relations 
%
\begin{enumerate}
	\item $\begin{aligned}[t] \chi_{\mt{AB}}(t) = -\Theta(t) \dv{t} \Phi_{\mt{AB}}(t) \end{aligned}$\hfill \refstepcounter{equation}(\theequation)\label{appeq: relation 1 between Phi and chi}
	\item $\begin{aligned}[t] \Phi_{\mt{AB}}(t = 0) = \chi_{\mt{AB}}(\omega = 0) \end{aligned}$\hfill \refstepcounter{equation}(\theequation)\label{appeq: relation 2 between Phi and chi}
	\item $\begin{aligned}[t] \Phi_{\mt{AB}}(\omega) = \frac{1}{i\omega}\big[\chi_{\mt{AB}}(\omega) - \chi_{\mt{AB}}(\omega = 0)\big]. \end{aligned}$\hfill \refstepcounter{equation}(\theequation)\label{appeq: relation 3 between Phi and chi}
\end{enumerate}
%
connecting the dynamical susceptibility $\chi_{\mt{AB}}$ with $\Phi_{\mt{AB}}$ are introduced.
In the following we want to proof these three relations.

The first one is easly gotten by derivating the Kubo relaxation function with respect to $t$ and comparing the result with the definition of the dynamical susceptibility \eqref{eq: dynamical susceptibilty}.
%
\begin{align}
	-\Theta(t) \dv{t} \Phi_{\mt{AB}}(t) = \frac{i}{\hbar} \Theta(t) \expval{\comm{\mt{A}_{\mt{I}}(t)}{\mt{B}_{\mt{I}}(0)}}_{0} = \chi_{\mt{AB}}(t)
\end{align}
%
The second relation is found with the aim of the Laplace transformation of the Kubo relaxation function.
%
\begin{align}
	\Phi_{\mt{AB}}(\omega) = \int\limits_{0}^{\infty} \dd{t} \Phi_{\mt{AB}}(t) e^{i\omega t}
	\label{appeq: Laplace transformation imaginary axis}
\end{align}
%
In this definition of the Laplace transformation compared to \eqref{eq: Laplace transformation real axis} we set $s = -i\omega$ which correspond to a rotation of $\frac{\pi}{2}$ of the definition space \todo{reference to a book of laplace transformation}.
Using \eqref{appeq: Laplace transformation imaginary axis} after setting $t = 0$ in \eqref{appeq: Kubo relaxation function} yield
%
\begin{align}
	\Phi_{\mt{AB}}(t=0) &= \frac{i}{\hbar} \lim\limits_{s \to 0} \int\limits_{0}^{\infty} \dd{\tau} \expval{\comm{\mt{A}_{\mt{I}}(\tau)}{\mt{B}_{\mt{I}}(0)}}_{0} e^{-s\tau}
	\notag \\
	\Leftrightarrow\ \Phi_{\mt{AB}}(t=0) &= \frac{i}{\hbar} \lim\limits_{\substack{s \to 0 \\ \omega \to 0}}\ \int\limits_{-\infty}^{\infty} \dd{\tau} \Theta(\tau) \expval{\comm{\mt{A}_{\mt{I}}(\tau)}{\mt{B}_{\mt{I}}(0)}}_{0} e^{i\omega \tau} e^{-s\tau}
	\notag \\
	\Leftrightarrow\ \Phi_{\mt{AB}}(t=0) &= \lim\limits_{\omega \to 0}\ \int\limits_{-\infty}^{\infty} \dd{\tau} \chi_{\mt{AB}}(\tau) e^{i\omega \tau}
	\notag \\
	\Leftrightarrow\ \Phi_{\mt{AB}}(t=0) &= \chi_{\mt{AB}}(\omega = 0),
\end{align}
%
where it is assumed the susceptibility is a good function in the sense they decay fast enough and the convergence generating faktor is negligible.
The third relation is computated with the aim of the first and second relation.
Therefore relation one is multiplied with $e^{i\omega t}$ and is integrated with respect to $t$.
%
\begin{align}
	\int\limits_{0}^{\infty} \dd{t} e^{i\omega t} \chi_{\mt{AB}}(t)  &= - \int\limits_{0}^{\infty} \dd{t} e^{i\omega t} \dv{t} \Phi_{\mt{AB}}(t)
	\notag \\
	\overset{\mt{PI}}{\Leftrightarrow}\ \chi_{\mt{AB}}(\omega) &= \eval{-e^{i\omega t} \Phi_{\mt{AB}}(t)}_{0}^{\infty} + i\omega \int\limits_{0}^{\infty} \dd{t} e^{i\omega t} \Phi_{\mt{AB}}(t)
	\notag \\
	\Leftrightarrow\ \chi_{\mt{AB}}(\omega) &= \Phi_{\mt{AB}}(t=0) + i\omega \Phi_{\mt{AB}}(\omega)
	\notag \\
	\Leftrightarrow\ \Phi_{\mt{AB}}(\omega) &= \frac{1}{i\omega} \big[\chi_{\mt{AB}}(\omega) - \chi_{\mt{AB}}(\omega=0)\big]
\end{align}
%
In the first step the right hand side is integrated by parts and in last step the first relation and \eqref{appeq: Laplace transformation imaginary axis} is used.
So the third relation gives us the dependence between the Kubo relaxation function and the dynamical susceptibility in frequency space. 