%
%
\chapter{Properties of the Kubo Relaxation Function}
\label{app:properties of the Kubo relaxation function}
%
%
The Kubo relaxation function 
%
\begin{align}
	\Phi_{\mt{AB}}(t) = \frac{i}{\hbar} \lim\limits_{s \to 0} \int\limits_{t}^{\infty} \dd{\tau} \expval{\comm{\mt{A}_{\mt{I}}(\tau)}{\mt{B}_{\mt{I}}(0)}}_{0} e^{-s\tau}.
	\label{appeq:Kubo relaxation function}
\end{align}
%
is introduced in section \ref{sec:kubo relaxation function} and the three relations below are still supposed.
%
\begin{enumerate}
	\item $\begin{aligned}[t] \chi_{\mt{AB}}(t) = -\Theta(t) \dv{t} \Phi_{\mt{AB}}(t) \end{aligned}$\hfill \refstepcounter{equation}(\theequation)\label{appeq:relation 1 between Phi and chi}
	\item $\begin{aligned}[t] \Phi_{\mt{AB}}(t = 0) = \chi_{\mt{AB}}(\omega = 0) \end{aligned}$\hfill \refstepcounter{equation}(\theequation)\label{appeq:relation 2 between Phi and chi}
	\item $\begin{aligned}[t] \Phi_{\mt{AB}}(\omega) = \frac{1}{i\omega}\big[\chi_{\mt{AB}}(\omega) - \chi_{\mt{AB}}(\omega = 0)\big]. \end{aligned}$\hfill \refstepcounter{equation}(\theequation)\label{appeq:relation 3 between Phi and chi}
\end{enumerate}
%
The evidence of these three relation is proven in this appendix.
The Kubo relaxation function is derivated with respect to the time $t$.
This yields immediatly the first relation, comparing the obtained derivative with the definition of the dynamical susceptibility \eqref{eq:dynamical susceptibilty}.
%
\begin{align}
	-\Theta(t) \dv{t} \Phi_{\mt{AB}}(t) = \frac{i}{\hbar} \Theta(t) \expval{\comm{\mt{A}_{\mt{I}}(t)}{\mt{B}_{\mt{I}}(0)}}_{0} = \chi_{\mt{AB}}(t)
	\label{eq:proof relation 1}
\end{align}
%
For the evidence of the second relation, the kubo relaxation function is evaluated for the time $t=0$.
The limit $\lim_{\omega\to0} \exp(i\omega \tau)$ is inserted and the lower limit of the integral is set to minus infinity, introducing the $\theta$-distribution $\theta(\tau)$.
In the last step equation \eqref{eq:proof relation 1} and the Laplace transformation are used.
%
\begin{align}
	\Phi_{\mt{AB}}(t=0) &= \frac{i}{\hbar} \lim\limits_{s \to 0} \int\limits_{0}^{\infty} \dd{\tau} \expval{\comm{\mt{A}_{\mt{I}}(\tau)}{\mt{B}_{\mt{I}}(0)}}_{0} e^{-s\tau}
	\notag \\
	\Leftrightarrow\ \Phi_{\mt{AB}}(t=0) &= \frac{i}{\hbar} \lim\limits_{\substack{s \to 0 \\ \omega \to 0}}\ \int\limits_{-\infty}^{\infty} \dd{\tau} \Theta(\tau) \expval{\comm{\mt{A}_{\mt{I}}(\tau)}{\mt{B}_{\mt{I}}(0)}}_{0} e^{i\omega \tau} e^{-s\tau}
	\notag \\
	\Leftrightarrow\ \Phi_{\mt{AB}}(t=0) &= \lim\limits_{\omega \to 0}\ \int\limits_{-\infty}^{\infty} \dd{\tau} \chi_{\mt{AB}}(\tau) e^{i\omega \tau} = \chi_{\mt{AB}}(\omega = 0)
\end{align}
%
Furthermore, the susceptibility is assumed to be good function in the sence of convergence and the limit respective to $s$ is hence taken with no doubt.
The thrid relation follows in combination of the first and second relation
Relation one is multiplied by $\exp(i\omega t)$ and is intgrated with respect to time $t$ from $0$ to $\infty$.
The definition of the Laplace transformation and integration by parts is used on the left and right hand side, respectively.
%
\begin{align}
	\int\limits_{0}^{\infty} \dd{t} \chi_{\mt{AB}}(t)\: e^{i\omega t} &= - \int\limits_{0}^{\infty} \dd{t} e^{i\omega t} \dv{t} \Phi_{\mt{AB}}(t)
	\notag \\
	\overset{\mt{PI}}{\Leftrightarrow}\ \chi_{\mt{AB}}(\omega) &= \eval{-e^{i\omega t} \Phi_{\mt{AB}}(t)}_{0}^{\infty} + i\omega \int\limits_{0}^{\infty} \dd{t} e^{i\omega t} \Phi_{\mt{AB}}(t)
	\notag \\
	\Leftrightarrow\ \Phi_{\mt{AB}}(\omega) &= \frac{1}{i\omega} \big[\chi_{\mt{AB}}(\omega) - \chi_{\mt{AB}}(\omega=0)\big]
\end{align}
%
