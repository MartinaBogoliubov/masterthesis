%
%
\chapter{Conclusion}
\label{ch:conclusion}
%
%
The spin-fermion-model, as suggested by \cite{Abanov&Chubukov&Schmalian}, describes metals and alloys at the vicinity of a quantum critical point.
Spin fluctuations are generated at the transition from the paramagnetic to the antiferromagnetic phase near absolute zero.
Large momentum $\vb{Q}$ is carried from these spin density waves and a interaction between fermions on the Fermi surface at $\vb{k}$ and $\vb{k}+\vb{Q}$ is therefore effected by them.
The corresponding free spin density propagator, depicted in equation \eqref{eq:undamped spin propagator}, is calculated by us and his periodicity is also shown using the equation of motion for Green functions.

The microscopic origin of spin fluctuations is due to permanent particle-hole excitations around the Fermi surface.
An interaction between spin density waves and partile-hole excitations is also present.
As a consequence of this interplay between spin fluctuations and fermions, damping is considered for the spin density waves.
The inverse lifetime of the spin fluctuation is corresponding to the full renormalized fermion bubble since they possess no own damping source
The damping term $\gamma|\omega_{n}|$ is computed using diagrammatic technique, where $\gamma$ is a damping constant and $\omega_{n}$ a Matsubara frequency (see equation \eqref{eq:damped spin propagator}).
The periodicity of the spin propagator is not changed due to damping.

Additionally, the conservation of momentum and current is proved in two cases: In the unperturbed spin-fermion-model and under consideration of umklapp scattering.
Heisenberg equation of motion is used to compute the time derivatives of both quantities.
In the unpertubated system the momentum is a conserved and the current is a non-conserved quantity (see equations \eqref{eq:time derivative momentum} and \eqref{eq:time derivative current}).
Considering umklapp scattering the momentum is also a non-conserved quantity (see equation \eqref{eq:time derivative momentum finite}), while current and its time derivative is unmodified.

In chapter \ref{ch:infinite conductivity} and \ref{ch:calculation} the static electrical conductivity is caluculated for two cases:
1) For a system with conserved momentum and non-conserved current and
2) for a system perturbed by umklapp scattering.
The memory-matrix-formalism, introduced by Mori \cite{Mori} and presented well by Forster \cite{Forster}, is used for our computation.
Autocorrelation functions are depending on a memory function $\mt{M}(z)$ and living in the Liouville space.
The latter is a vector space that has operators acting on a usual quantum mechanical Hilbert sapce as basis vectors.
The history of the autocorrelation function is described by the memory function which decays exponential.
This is not the case for conserved quantities and the time evolution of correlation functions is therefore described correct also in the limit $t\to\infty$ or $\omega\to0$.
In chapter \ref{ch:infinite conductivity} it is shown that the static conductivity attains a infinite value in the case of conserved momentum and non-conserved current.
Our main calculation is presented in chapter \ref{ch:calculation}.
Similar to the treatment of Patel and Sachdev \cite{Patel&Sachdev}, the static electrical conductivity is computed for the spin-fermion-model.
Our model is modified by an anisotropic parabolical fermionic dispersion and umklapp scattering as pertubation.
The static conductivity $\sigma_{\mt{dc}}$ is proportional to the susceptibility $\chi_{\mt{JP}}$ and the inverse Green function $\mathcal{G}_{\dot{\mt{P}}\dot{\mt{P}}}(\vb{k},z)$.
$\chi_{\mt{JP}}$ is exactly calculated with the result of temperature independence.
For the Green function the ontained integral out of the diagrammatic technique is not exactly solvable.
Therefore the certain case $\vb{G}-\vb{Q}_{1} = \vb{Q}_{2} = 0$ is observed since this one possesses a strong singularity and governs therefore the behaviour.
In this case a temperature dependence of $T^{1-4/z}$ is found, where $z$ is a critical exponent.
The resistance is proportional to this Green function and possesses therefore the same temperature dependence.
For the usually used values of $z$ ($z=1$ and $z=2$) the tempreature dependence of the resistance is determined to $T^{-3}$ and $T^{-1}$, respectivily.
The usual values for $z$ are $z=1$ or $z=2$ for higher or lower temperature,respectively, at the vicinity of the quantum critical point.
In both cases the temperature dependence is highly divergent and disagree with the expected linear temperature dependence of the resistance shown in the measurements \cite{Loehneysen}.

The computed temperature dependence is surely to divergent as to be correct.
In the diagrammatic technique the time evolution operatore containing the spin-fermion interaction $\mt{H}_{\Psi\Phi}$ is only expanded up to the zeroth order.
In our opinion, the divergent temperature dependence of the resistance is adjust by consideration of higher orders in the series expansion of the time evolution operator.
The question if umklapp scattering causes a linear temperature dependence of the resistance in heavy-fermion system, like $\mt{CeCu}_{5.9}\mt{Au}_{0.1}$ and $\mt{CeCu}_{5.8}\mt{Ag}_{0.2}$, is still unanswered.
The calculation considering higher order terms is still an interesting task for the future.














