%
%
%
\chapter{Time Derivative of momentum P and current J}
\label{appch:time derivative P and J}
%
%
%
The time derivative of the momentum and current operator are designated for the case of non-pertubation and pertubation in chapter \ref{ch:spin fermion model}.
In this appendix, the computation is outlined of them.
Our further approach is to compute Hamiltonian, momentum and current operator, using the energy-momentum-tensor and current-tensor \cite{Iliev}.
The time derivative of momentum and current is then calculated, using the Heisenberg equation of motion.
In the end, umklapp scattering is observed and the effect to momentum and current are discussed.

Our starting point is the Lagrangian $\mathcal{L} = \mathcal{L}_{\Psi} + \mathcal{L}_{\Phi} + \mathcal{L}_{\Psi\Phi}$, as suggested in \cite{Patel&Sachdev} of the spin-fermion-model.
In Matsubara time $\tau = it$, the single Lagrangians are given by
%
\begin{align}
	\mathcal{L}_{\Psi} = 
		\vb{\Psi}^{\dag}(\vb{x},\tau)
		\twoTwoMatrix{\sigma_{0} (\partial_{\tau}+\mu_{0}+\xi_{a})}{0}{0}{\sigma_{0} (\partial_{\tau}+\mu_{0}+\xi_{b})}
		\vb{\Psi}(\vb{x},\tau)
\end{align}
%
\begin{align}
	\mathcal{L}_{\Phi} &= 
		\frac{1}{2} \big[\partial_{i}\Phi_{\mu}(\vb{x},\tau)\big] \big[\partial_{i}\Phi_{\mu}(\vb{x},\tau)\big] 
		+ 
		\frac{\epsilon}{2} \big[\partial_{\tau}\Phi_{\mu}(\vb{x},\tau)\big] \big[\partial_{\tau}\Phi_{\mu}(\vb{x},\tau)\big] 
		\notag \\&+
		\frac{u}{6} \Big[\Phi_{\mu}(\vb{x},\tau) \Phi_{\mu}(\vb{x},\tau) + \frac{3}{g}\Big]^{2}
\end{align}
%
\begin{align}
	\mathcal{L}_{\Psi\Phi} =
		\lambda\Phi_{\mu}(\vb{x},\tau) \Big[\Psi_{\mt{a}}^{\dag}(\vb{x},\tau) \sigma_{\mu} \Psi_{\mt{b}}(\vb{x},\tau) + \Psi_{\mt{b}}^{\dag}(\vb{x},\tau) \sigma_{\mu} \Psi_{\mt{a}}(\vb{x},\tau)\Big]
\end{align}
%
Here $\vb{\Psi} = (\Psi_{\mt{a}}, \Psi_{\mt{b}})$ is a vector containing the two-component spinors $\Psi_{\mt{a}}$, $\Psi_{\mt{b}}$.
$\Phi_{\mu}$ is the three component bosnoic field operator and describes the spin fluctuations in the spin-fermion-model.
Since the spin fluctuations arise at the phase transition, $\Phi_{\mu}$ is order parameter as well.
The chemical potential is denoted with $\mu_{0}$ and the Pauli matrix are labeled with $\sigma_{0}$ and $\sigma_{\mu}$.
The abbreviation $\epsilon = v_{\mt{S}}^{-2}$ is introduced, where $v_{\mt{S}}$ is the spin velocity.
In the Lagrangian $\mathcal{L}_{\Phi}$, the term in the second line is neglectable in the low-energy theory of the spin fermion model.
The anisotropic parabolical dispersion relations $\xi_{\mt{a}}$ and $\xi_{\mt{b}}$ are given by 
%
\begin{align}
	\xi_{\mt{a}} = \frac{\partial_{x}^{2}}{2m_{1}} + \frac{\partial_{y}^{2}}{2m_{2}} \qquad \xi_{\mt{b}} = \frac{\partial_{x}^{2}}{2m_{2}} + \frac{\partial_{y}^{2}}{2m_{1}}
\end{align}
%
The Hamiltonian for umklapp scattering is assumed to be 
%
\begin{align}
	\mathcal{H}_{\mt{umklapp}}= \mt{J}(\vb{R}) \Phi_{\mu}(\vb{x},\tau) \Phi_{\mu}(\vb{x},\tau)
\end{align}
%
where $\mt{J}(\vb{R})$ is a coupling parameter and $\vb{R}$ is a lattice vector.
In a first step, the Hamiltonian for the unpertubated spin-fermion-model is calulated using the energy-momentum-tensor and current-tensor \cite{Iliev}.
%
\begin{align}
	\mathcal{T}_{\mu\nu} = 
		\frac{1}{2} \sum\limits_{\zeta_{n}} \bigg[ 
		\Big(\frac{\partial\mathcal{L}}{\partial(\partial_{\mu}\zeta_{n})}\Big) (\partial_{\nu}\zeta_{n}) 
		+
		(\partial_{\nu}\zeta_{n})^{\dag} \Big(\frac{\partial\mathcal{L}}{\partial(\partial_{\mu}\zeta_{n})}\Big)^{\dag}
		\bigg]
		-\eta_{\mu\nu}\mathcal{L}
\end{align}
%
%
\begin{align}
	\mathcal{J}_{\mu} = -i \sum\limits_{\{\zeta_{n}\}} \epsilon(\zeta_{n}) \bigg[
		\Big(\frac{\partial\mathcal{L}}{\partial(\partial_{\mu}\zeta_{n})}\Big) \zeta_{n}
		-
		\zeta_{n}^{\dag} \Big(\frac{\partial\mathcal{L}}{\partial(\partial_{\mu}\zeta_{n})}\Big)^{\dag}
		\bigg],
\end{align}
%
Here, $\zeta_{n}$ is the set of all contributing operator fields, which is in our case $\{\Psi_{\mt{a}},\Psi_{\mt{b}}, \Phi_{\mu}\}$.
The function $\epsilon(\zeta_{n})$ attains the values 0 of 1 in the case of real or complex fields, respectively.
$\eta_{\mu\nu}$ is the Minkowski metric, where the definition $(1,-1,-1,-1)$ is used here.
In chapter \ref{ch:spin fermion model}, the concept of hot spots is introduced.
Fermions on different Fermi surfaces interact via spin fluctations, described by $\Phi_{\mu}$.
The bosonic field operators are supposed to be real quantities, since the interaction is assumed to be on the hot spots and not in a vicinity regime around them.
Furthermore, the spatial derivative in $\mathcal{L}_{\Psi}$ is rewriten. using relation $\Psi^{\dag} (\partial_{i}^{2} \Psi) = -(\partial_{i} \Psi^{\dag}) (\partial_{i} \Psi)$.
This relation is evaluated, integrating the left hand side by parts.

The (00)-component of the energy-momnetum-tensor represents the Hamiltonian of a system.
This quantity is computed in the following.
%
\begin{align}
	\mathcal{H} &= 
		\frac{1}{2} \sum\limits_{\zeta_{n}} \bigg[
		\Big(\frac{\partial\mathcal{L}}{\partial(\partial_{\tau}\zeta_{n})}\Big) (\partial_{\tau}\zeta_{n}) 
		-
		(\partial_{\tau}\zeta_{n}^{\dag}) \Big(\frac{\partial\mathcal{L}}{\partial(\partial_{\tau}\zeta_{n})}\Big)^{\dag}
		\bigg]
		-
		\mathcal{L}
		\notag \\
	\Leftrightarrow\ \mathcal{H} &= 
		\psi_{\mt{a}}^{\dag} (\partial_{\tau}\psi_{\mt{a}}) 
		+
		\psi_{\mt{b}}^{\dag} (\partial_{\tau}\psi_{\mt{b}}) 
		+
		\epsilon  (\partial_{\tau} \Phi_{\mu}) (\partial_{\tau} \Phi_{\mu}) 
		-
		\mathcal{L}
	\notag \\
	\Leftrightarrow\ \mathcal{H} &= 
		-
		\psi_{\mt{a}}^{\dag}(\vb{x},\tau) \big(\frac{\partial_{x}^{2}}{2m_{1}} + \frac{\partial_{y}^{2}}{2m_{2}} + \mu_{0}\big) \psi_{\mt{a}}(\vb{x},\tau)
		\notag \\&
		- 
		\psi_{\mt{b}}^{\dag}(\vb{x},\tau) \big(\frac{\partial_{x}^{2}}{2m_{2}} + \frac{\partial_{y}^{2}}{2m_{1}} + \mu_{0}\big) \psi_{\mt{b}}(\vb{x},\tau)
		\notag \\&
		- 
		\frac{1}{2} \big(\partial_{i} \Phi_{\mu}(\vb{x},\tau)\big) \big(\partial_{i} \Phi_{\mu}(\vb{x},\tau)\big)
		+ 
		\frac{1}{2\epsilon} \pi_{\mu}(\vb{x},\tau) \pi_{\mu}(\vb{x},\tau)
		\notag \\&
		-	
		\lambda \Phi_{\mu}(\vb{x},\tau) \big(\psi_{\mt{a}}^{\dag}(\vb{x},\tau) \sigma_{\mu} \psi_{\mt{b}}(\vb{x},\tau) + \psi_{\mt{b}}^{\dag}(\vb{x},\tau) \sigma_{\mu} \psi_{\mt{a}}(\vb{x},\tau)\big)
\end{align}
%
Here, the usual time derivative is transformed into the Matsubara time, using $\partial_{t} = -i\partial_{\tau}$.
The Hermitain of the $\tau$-derivative is given by $(\partial_{\tau})^{\dag} = -\partial_{\tau}$, which is also used.
The canonical momentum of $\Phi_{\mu}$, given by $\pi_{\mu}(\vb{x},\tau) = \epsilon \partial_{\tau} \Phi_{\mu}(\vb{x},\tau)$, is used in the last step.
The $\mathcal{T}_{0j}$-component is identified with the $j$-component of the momentum operator, which is calculated as next.
%
\begin{align}
	\mathcal{P}_{j} &= 
		\frac{-i}{2} \sum\limits_{\zeta_{n}} \bigg[ 
		\Big(\frac{\partial\mathcal{L}}{\partial(\partial_{\tau}\zeta_{n})}\Big) (\partial_{j}\zeta_{n}) 
		+
		(\partial_{j}\zeta_{n})^{\dag} \Big(\frac{\partial\mathcal{L}}{\partial(\partial_{\tau}\zeta_{n})}\Big)^{\dag}
		\bigg]
		-
		\eta_{0j}\mathcal{L}
		\notag \\
	\Leftrightarrow\ \mathcal{P}_{j} &= 
		\frac{-i}{2} \bigg[
			\psi_{\mt{a}}^{\dag}(\vb{x},\tau) \big(\partial_{j} \psi_{\mt{a}}(\vb{x},\tau)\big)
			- 
			\big(\partial_{j} \psi_{\mt{a}}^{\dag}(\vb{x},\tau)\big) \psi_{\mt{a}}(\vb{x},\tau)
			\notag \\&
			+
			\psi_{\mt{b}}^{\dag}(\vb{x},\tau) \big(\partial_{j} \psi_{\mt{b}}(\vb{x},\tau)\big)
			- 
			\big(\partial_{j} \psi_{\mt{b}}^{\dag}(\vb{x},\tau)\big) \psi_{\mt{b}}(\vb{x},\tau)
		\bigg]
		- 
		i\pi_{\mu}(\vb{x},\tau)) \big(\partial_{j} \phi_{\mu}(\vb{x},\tau)\big)
	\notag \\
	\Rightarrow\ \vb{\mathcal{P}} &= 
		\frac{-i}{2} \bigg[
			\vb{\Psi}^{\dag}(\vb{x},\tau) \big(\nabla \vb{\Psi}(\vb{x},\tau)\big)
			- 
			\big(\nabla \vb{\Psi}^{\dag}(\vb{x},\tau)\big) \vb{\Psi}(\vb{x},\tau)
		\bigg]
		-
		i \pi_{\mu}(\vb{x},\tau) \big(\nabla \phi_{\mu}(\vb{x},\tau)\big)
\end{align}
%
The Hermitian of the $\tau$-derivative and the canonical momentum is again used at this computation.
The two-component vector $\vb{\Psi}$ is inserted to write the vector components as vector.
Now, the $x$-component of the current is caluclated.
%
\begin{align}
	\mathcal{J}_{x} &= 
		-i \sum\limits_{\zeta_{n}} \epsilon(\zeta_{n}) \bigg[
			\Big(\frac{\partial\mathcal{L}}{\partial(\partial_{x}\zeta_{n})}\Big) \zeta_{n}
			-
			\zeta_{n}^{\dag} \Big(\frac{\partial\mathcal{L}}{\partial(\partial_{x}\zeta_{n})}\Big)^{\dag}
		\bigg]
	\notag \\
	\Leftrightarrow\ \mathcal{J}_{x} &=
		\frac{-i}{2} \bigg[
		\frac{1}{m_{1}} \Big[
			\big(\partial_{x} \psi_{\mt{a}}^{\dag}(\vb{x},\tau)\big) \psi_{\mt{a}}(\vb{x},\tau)
			-
			\psi_{\mt{a}}^{\dag}(\vb{x},\tau) \big(\partial_{x} \psi_{\mt{a}}(\vb{x},\tau)\big)
		\Big]
		\notag \\&
		+
		\frac{1}{m_{2}} \Big[
			\big(\partial_{x} \psi_{\mt{b}}^{\dag}(\vb{x},\tau)\big) \psi_{\mt{b}}(\vb{x},\tau)
			-
			\psi_{\mt{b}}^{\dag}(\vb{x},\tau) \big(\partial_{x} \psi_{\mt{b}}(\vb{x},\tau)\big)
		\Big]
		\bigg]
\end{align}
%
The $x$-component of the current is compted analogical, where an interchange of the mass $(m_1 \leftrightarrow m_2)$ is the only difference. 
These three quantities are now transformed into momentum space, since all of them contains spatial derivatives.
This is obstructive in the later caluclation of the time derivative of P and J, since commutator relations of fermionic and bosonic fields are therefore used.
%
\begin{align}
	\mt{H} &= 
	 	\int_{\vb{k}}
	 	\Bigg[ \sum\limits_{\alpha} \epsilon_{\alpha}(\vb{k}) \psi_{\alpha}^{\dag}(\vb{k},\tau) \psi_{\alpha}(\vb{k},\tau)
		-
		\frac{\vb{k}^{2}}{2} \Phi_{\mu}(\vb{k},\tau) \Phi_{\mu}(-\vb{k},\tau)
		+
		\frac{1}{2\epsilon} \pi_{\mu}(\vb{k},\tau) \pi_{\mu}(-\vb{k},\tau)
		\notag \\ &
		-
		\lambda \int_{\vb{q}} \Phi_{\mu}(\vb{k}-\vb{q},\tau)
		\bigg[
			\psi_{\mt{a}}^{\dag}(\vb{k},\tau) \sigma_{\mu} \psi_{\mt{b}}(\vb{q},\tau)
			+
			\psi_{\mt{b}}^{\dag}(\vb{k},\tau) \sigma_{\mu} \psi_{\mt{a}}(\vb{q},\tau)
		\bigg]
		\Bigg]
\end{align}
%
%
\begin{align}
	\mt{P}_{j} &= \int_{\vb{k}}
		k_{j} \Bigg[ \sum\limits_{\alpha} \psi_{\alpha}^{\dag}(\vb{k},\tau) \psi_{\alpha}(\vb{k},\tau)
	 	-
	 	\pi_{\mu}(\vb{k},\tau)
	 	\Phi_{\mu}(-\vb{k},\tau)
	\Bigg]
\end{align}
%
%
\begin{align}
	\mt{J}_{x} &= - \int_{\vb{k}} \Bigg[
		\frac{k_{x}}{m_{1}}
		\psi_{\mt{a}}^{\dag}(\vb{k},\tau)
		\psi_{\mt{a}}(\vb{k},\tau)
		+
		\frac{k_{x}}{m_{2}}
		\psi_{\mt{b}}^{\dag}(\vb{k},\tau)
		\psi_{\mt{b}}(\vb{k},\tau)
	\Bigg]
\end{align}
%
The sum over $\alpha$ summarizes over the two species of fermions a and b.
Their dispersion relations are given by $\epsilon_{\mt{a}}(\vb{k}) = \frac{k_{x}^{2}}{2m_{1}} + \frac{k_{y}^{2}}{2m_{2}} - \mu_{0}$ und $\epsilon_{\mt{b}}(\vb{k}) = \frac{k_{x}^{2}}{2m_{2}} + \frac{k_{y}^{2}}{2m_{1}} - \mu_{0}$ in momentum space.
All integrals are extended over the first Brillouin zone.

Now, the time derivative of momentum P and current J is computed.
The Heisenberg equation of motion is the usual way to calculate the derivative of an operator in quantum mechanics.
The simplicity of this method is, that only the commutator between the Hamiltonian and the observed operator has to be computed.
For the further approach, the commutator relations for bosonic and fermionic fields are required.
Beside these two ones, all other commutator relations yield zero.
%
\begin{align}
	\acomm{\psi_{\mt{\alpha}}(\vb{k},\tau)}{\psi_{\mt{\beta}}^{\dag}(\vb{p},\tau)} &= (2\pi)^{2} \delta_{\mt{\alpha}\mt{\beta}} \delta(\vb{p}-\vb{k})
	\\
	\comm{\Phi_{\mu}(\vb{k},\tau)}{\pi_{\lambda}(\vb{p},\tau)} &= (2\pi)^{2} \delta_{\mu\lambda} \delta(\vb{p}+\vb{k})
\end{align}
%
At first the commutator between the Hamiltonian H and the $x$-component of P is computed.
All integrals are extended over the first Brillouin zone, while the sums over greek letters runs over the fermionic species a and b.
The condition $\alpha \neq \beta$ means that in this case both greek latters has to be different.
The commutator has to be transformed into component representation, if Pauli matricies $\sigma_{\mu}$ connects to fermionic field operators.
The above commutator relations are also valid in this representation.
%
\begin{align}
	\comm{\mt{H}}{\mt{P}_{x}} &= 
		\int_{\vb{k}} \int_{\vb{p}}
		p_{x}
		\sum\limits_{\alpha,\beta} 
		\epsilon_{\alpha}(\vb{k})
		\comm{\psi_{\alpha}^{\dag}(\vb{k},\tau) \psi_{\alpha}(\vb{k},\tau)}{\psi_{\beta}^{\dag}(\vb{k},\tau) \psi_{\beta}(\vb{k},\tau)}
		\notag \\&
		+
		\int_{\vb{k}} \int_{\vb{p}}
		p_{x}
		\bigg[
			\frac{k^{2}}{2}
			\comm{\Phi_{\mu}(\vb{k},\tau) \Phi_{\mu}(-\vb{k},\tau)}{\pi_{\lambda}(\vb{p},t) \Phi_{\lambda}(-\vb{p},t)}
			\notag \\& \hspace{2cm}
			-
			\frac{1}{2\epsilon}
			\comm{\pi_{\mu}(\vb{k},\tau) \pi_{\mu}(-\vb{k},\tau)}{\pi_{\lambda}(\vb{p},t) \Phi_{\lambda}(-\vb{p},t)}
		\bigg]
		\notag \\&
		-
		\lambda
		\int_{\vb{k}} \int_{\vb{p}} \int_{\vb{q}}
		p_{x}
		\Phi_{\mu}(\vb{k}-\vb{q},\tau)
		\sum\limits_{\alpha\neq\beta}
		\sum\limits_{\gamma}
		\comm{\psi_{\alpha}^{\dag}(\vb{k},\tau) \sigma_{\mu} \psi_{\beta}(\vb{q},\tau)}{\psi_{\gamma}^{\dag}(\vb{p},t) \psi_{\gamma}(\vb{p},t)}
		\notag \\&
		+
		\lambda
		\int_{\vb{k}} \int_{\vb{p}} \int_{\vb{q}}
		p_{x}
		\sum\limits_{\alpha\neq\beta}
		\psi_{\alpha}^{\dag}(\vb{k},\tau) \sigma_{\mu} \psi_{\beta}(\vb{q},\tau)
		\comm{\Phi_{\mu}(\vb{k}-\vb{q},\tau)}{\pi_{\lambda}(\vb{p},t) \Phi_{\lambda}(-\vb{p},t)}
	\notag \\
	\Leftrightarrow\ \comm{\mt{H}}{\mt{P}_{x}} &=	
		-
		\lambda
		\int_{\vb{k}} \int_{\vb{q}}
		(q_{x}-k_{x})
		\Phi_{\mu}(\vb{k}-\vb{q},\tau)
		\sum\limits_{\alpha\neq\beta}
		\psi_{\alpha}^{\dag}(\vb{k},\tau) \sigma_{\mu} \psi_{\beta}(\vb{q},t)
		\notag \\&
		+
		\lambda
		\int_{\vb{k}} \int_{\vb{q}}
		(q_{x}-k_{x})
		\Phi_{\mu}(\vb{k}-\vb{q},t)
		\sum\limits_{\alpha\neq\beta}
		\psi_{\alpha}^{\dag}(\vb{k},\tau) \sigma_{\mu} \psi_{\beta}(\vb{q},\tau)
	\notag \\
	\Leftrightarrow\ \comm{\mt{H}}{\mt{P}_{x}} &= 0
\end{align}
%
This result is also obtained for the $y$-direction of the momentum.
As consequence, the time derivative of momentum is zero, $\dot{\mt{P}} = 0$.
In the spin-fermion-model, described by the Hamiltonian H, momentum is a conserved quantity.
This implies a unbroken translation symmetry.
At next, the time derivative of the current in $x$-direction is calculated.
The notation is equal to this used by the momentum above.
%
\begin{align}
	\comm{H}{J_{x}} &=
		\int_{\vb{k}} \int_{\vb{p}}
		\frac{p_{x}}{m_{1}}
		\sum\limits_{\alpha} 
		\epsilon_{\alpha}(\vb{k})
		\comm{
			\psi_{\alpha}^{\dag}(\vb{k},\tau) 
			\psi_{\alpha}(\vb{k},\tau)
		}{	
			\psi_{\mt{a}}^{\dag}(\vb{p},\tau)	
			\psi_{\mt{a}}(\vb{p},\tau)
		}
		\notag \\&
		+
		\int_{\vb{k}} \int_{\vb{p}}
		\frac{p_{x}}{m_{2}}
		\sum\limits_{\alpha} 
		\epsilon_{\alpha}(\vb{k})
		\comm{
			\psi_{\alpha}^{\dag}(\vb{k},\tau) 
			\psi_{\alpha}(\vb{k},\tau)
		}{	
			\psi_{\mt{b}}^{\dag}(\vb{p},\tau)
			\psi_{\mt{b}}(\vb{p},\tau)
		}
		\notag \\&
		+
		\lambda
		\int_{\vb{k}} \int_{\vb{p}} \int_{\vb{q}}
		\Phi_{\mu}(\vb{k}-\vb{q},\tau)
		\frac{p_{x}}{m_{1}}
		\sum\limits_{\alpha\neq\beta}
		\comm{
			\psi_{\alpha}^{\dag}(\vb{k},\tau) 
			\sigma_{\mu} 
			\psi_{\beta}(\vb{q},t)
		}{
			\psi_{\mt{a}}^{\dag}(\vb{p},\tau)	
			\psi_{\mt{a}}(\vb{p},\tau)
		}
		\notag \\&
		+
		\lambda
		\int_{\vb{k}} \int_{\vb{p}} \int_{\vb{q}}
		\Phi_{\mu}(\vb{k}-\vb{q},\tau)
		\frac{p_{x}}{m_{2}}
		\sum\limits_{\alpha\neq\beta}
		\comm{
			\psi_{\alpha}^{\dag}(\vb{k},\tau) 
			\sigma_{\mu} 
			\psi_{\beta}(\vb{q},t)
		}{
			\psi_{\mt{b}}^{\dag}(\vb{p},\tau)
			\psi_{\mt{b}}(\vb{p},\tau)
		}
	\notag \\
	\Leftrightarrow \comm{H}{J_{x}} &=
		\lambda
		\int_{\vb{k}} \int_{\vb{q}}
		\Phi_{\mu}(\vb{k}-\vb{q},\tau) 
		\notag \\&
		\times \Big[
			\Big(\frac{q_{x}}{m_{1}} - \frac{k_{x}}{m_{2}}\Big)
			\psi_{\mt{b}}^{\dag}(\vb{k},\tau) \sigma_{\mu} \psi_{\mt{a}}(\vb{q},\tau)
			+
			\Big(\frac{q_{x}}{m_{2}} - \frac{k_{x}}{m_{1}}\Big)
			\psi_{\mt{a}}^{\dag}(\vb{k},\tau) \sigma_{\mu} \psi_{\mt{b}}(\vb{q},\tau)
		\Big]
\end{align}
%
In the investigated spin-fermion-model, the time derivative of the current is non-zero and as a consequence the current is unconserved.
This property of the model is important for the calculation of the static conductivity.
The effect of umklapp scatering respective the time derivative of P and J is following computed.
The Hamiltonian $\mt{H}_{\mt{umklapp}}$ is given by
%
\begin{align}
	\mt{H}_{\mt{umklapp}}= \sum\limits_{\vb{G}} \mt{J}_{\vb{G}} \int_{\vb{k}} \Phi_{\mu}(\vb{k},\tau) \Phi_{\mu}(-\vb{k}+\vb{G},\tau)
\end{align}
%
in momentum space.
The intergal is extended over the first Brillouin zone and the sum runs over all reciprocal lattice vectors.
The quantities P and J are not altered considering Umklapp scattering, since momentum and current only emerged, if the Hamiltonian possesses time or spatial derivatives, respectiviely.
Both are not existing in the Hamiltonian $\mt{H}_{\mt{umklapp}}$.
Nevertheless, the time derivative of P and J are possible modified.
The contribution of umklapp scattering to the latter is zero. 
The field operators of the current are of fermionic nature, while these of the Hamiltonian $\mt{H}_{\mt{umklapp}}$ are of bosonic nature.
The time derivative of P is in contrast non-zero.
%
\begin{align}
	\dot{P}_{j} &= -i \sum\limits_{G} \mt{J}_{\vb{G}} 
		\int_{\vb{k}} \int_{\vb{p}}
		p_{j}
		\comm{\Phi_{\mu}(k) \Phi_{\mu}(-k+G)}{\pi_{\lambda}(p) \Phi_{\lambda}(-p)}
	\notag \\
	\dot{P}_{j} &= -i \sum\limits_{G} \mt{J}_{\vb{G}} 
		\int_{\vb{k}} \int_{\vb{p}} 
		p_{j} \bigg[
			\Phi_{\mu}(k) \comm{\Phi_{\mu}(-k+G)}{\pi_{\lambda}(p)} \Phi_{\lambda}(-p)
			\notag \\ &\hspace{4.05cm}+
			\comm{\Phi_{\mu}(k)}{\pi_{\lambda}(p)} \Phi_{\mu}(-k+G) \Phi_{\lambda}(-p)
		\bigg]
	\notag \\
	\dot{P}_{j} &= -i \sum\limits_{G} \mt{J}_{\vb{G}} 
		\int_{\vb{k}} 
		\bigg[
			(k_{j} - G_{j}) \Phi_{\mu}(k) \Phi_{\mu}(-k+G)
			-
			k_{j} \Phi_{\mu}(-k+G) \Phi_{\mu}(k)
		\bigg]
	\notag \\
	\dot{P}_{j} &= i \sum\limits_{G} \mt{J}_{\vb{G}} 
		\int_{\vb{k}} G_{j} \Phi_{\mu}(k) \Phi_{\mu}(-k+G)
	\label{appeq:time derivative momentum}
\end{align}
%
Considering umklapp scattering the time derivative of P is determined to an finite value.
As a consequence the momentum is not conserved any longer.
The characteristic of the considered pertubation to change momentum and leave current unchange is important for the compuation of the static electrical conductivity.










































































