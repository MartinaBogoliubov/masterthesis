%
%
%
\chapter{Computation of the spin density wave propagator}
\label{ch: propagator}
%
%
%
\todo{Einleitung zum Kapitel schreiben}
%
%
\section{The free propagator of spin density waves}
\label{sec: free propagator}
%
%
The dynamic of spin density waves without interaction is described by the Hamiltonian $\mt{H}_{\Phi}$, introduced in chapter \ref{ch: spin fermion model}.
The equation of motion of the Green's functions are one naturally, simple and elegant method to compute a free propagtor of an aribtrary typ of particle.
The equation of motion in frequency spce is the first item of an infinte algebraic equation chain.
Mostly the initial Green function appears lately in the second item of the chain which is resulting in an solvable algebraic equtaion system.
For a detailed deviation the book of Elk and Gasser \cite{Elk&Gasser} is recommended.
In the case of the free spin density waves the first item of the chain is
%
\begin{align}
	\omega \green{\Phi_{\mu}(\vb{k}+\vb{G},t)}{\Phi_{\lambda}(\vb{k}'+\vb{G}',t')}_{\omega} &= 
		\expval{\comm{\Phi_{\mu}(\vb{k}+\vb{G},t)}{\Phi_{\lambda}(\vb{k}'+\vb{G}',t')}}
		\notag \\ &+
		\green{\comm{\Phi_{\mu}(\vb{k}+\vb{G},t)}{\mt{H}_{\Phi}}}{\Phi_{\lambda}(\vb{k}'+\vb{G}',t')}_{\omega},
	\label{eq: first item of equation chain}
\end{align}
%
where the double angle brackets denotes the Green function of spin density waves and the index $\omega$ means that the Green function has to be considered in the frequency space.
The simplicity and advantage of the method instead of other ones is that only the commutator relations of the (field) operators are needed.
The bosonic commutator relations are given in equation (\dots\todo{link zu bosonischen Vertauschungsrelationen}) and the Hamiltonian is given by equation (\dots\todo{Link zu $H_{\Phi}$}).
The inhomogeneity is trivially zero and the commutator in the second term yields
%
\begin{align}
	\comm{\Phi_{\mu}(\vb{k}+\vb{G},t)}{\mt{H}_{\Phi}} &= 
		-\frac{1}{2\epsilon} 
		\sum\limits_{\vb{P}} 
		\int_{\vb{p}}
		\comm{\Phi_{\mu}(\vb{k}+\vb{G},t)}{\pi_{\nu}(\vb{p}+\vb{P},t_{1}) \pi_{\nu}(-\vb{p}-\vb{P},t_{1})}
	\notag \\
	\Leftrightarrow\ \comm{\Phi_{\mu}(\vb{k}+\vb{G},t)}{\mt{H}_{\Phi}} &= 
		-\frac{1}{2\epsilon} 
		\sum\limits_{\vb{P}} 
		\int_{\vb{p}} \bigg[
			\pi_{\nu}(\vb{p}+\vb{P},t_{1}) \comm{\Phi_{\mu}(\vb{k}+\vb{G},t)}{\pi_{\nu}(-\vb{p}-\vb{P},t_{1})}
			\notag \\& \hspace{2cm}
			+
			\comm{\Phi_{\mu}(\vb{k}+\vb{G},t)}{\pi_{\nu}(\vb{p}+\vb{P},t_{1})} \pi_{\nu}(-\vb{p}-\vb{P},t_{1})
		\bigg]
	\notag \\
	\Leftrightarrow\ \comm{\Phi_{\mu}(\vb{k}+\vb{G},t)}{\mt{H}_{\Phi}} &= 
		-\frac{i}{\epsilon} \pi_{\mu}(\vb{k}+\vb{G},t)
\end{align}
%
where in the beginning the sum over $\lambda$ is implied.
Equation \eqref{eq: first item of equation chain} yields
%
\begin{align}
	\omega \green{\Phi_{\mu}(\vb{k}+\vb{G},t)}{\Phi_{\lambda}(\vb{k}'+\vb{G}',t')}_{\omega} &= 
		-\frac{i}{\epsilon} \green{\pi_{\mu}(\vb{k}+\vb{G},t)}{\Phi_{\lambda}(\vb{k}'+\vb{G}',t')}_{\omega}
\end{align}
%
where on the right hand side a new Green function appears.
Nobody bars us doing the exactly same procedure for this one as for the initial Green function.
%
\begin{align}
	\omega \green{\pi_{\mu}(\vb{k}+\vb{G},t)}{\Phi_{\lambda}(\vb{k}'+\vb{G}',t')}_{\omega} &= 
		\expval{\comm{\pi_{\mu}(\vb{k}+\vb{G},t)}{\Phi_{\lambda}(\vb{k}'+\vb{G}',t')}}
		\notag \\ &+
		\green{\comm{\pi_{\mu}(\vb{k}+\vb{G},t)}{\mt{H}_{\Phi}}}{\Phi_{\lambda}(\vb{k}'+\vb{G}',t')}_{\omega}
	\label{eq: second item of equation chain}
\end{align}
%
The inhomogeneity of the equations is trivially, too, but in this case yields something differant to zero.
%
\begin{align}
	\comm{\pi_{\mu}(\vb{k}+\vb{G},t)}{\Phi_{\lambda}(\vb{k}'+\vb{G}',t')} = -i (2\pi)^{2} \sum\limits_{\vb{G}} \delta(\vb{k}'+\vb{k}) \delta_{-\vb{G}',\vb{G}} \delta_{\mu,\lambda}
\end{align}
%
Equally the commutator can be computed with the same method.
%
\begin{align}
	\comm{\pi_{\mu}(\vb{k}+\vb{G},t)}{\mt{H}_{\Phi}} &= 
		-\sum\limits_{\vb{P}} 
		\int_{\vb{p}}
		\frac{(\vb{p}+\vb{P})^{2} + r}{2}
		\notag \\ &\times
		\comm{\pi_{\mu}(\vb{k}+\vb{G},t)}{\Phi_{\nu}(\vb{p}+\vb{P},t_{1}) \Phi_{\nu}(-\vb{p}-\vb{P},t_{1})}
	\notag \\
	\Leftrightarrow\ \comm{\pi_{\mu}(\vb{k}+\vb{G},t)}{\mt{H}_{\Phi}} &= 
		i(2\pi)^{2} \sum\limits_{\vb{P}} 
		\int_{\vb{p}}
		\frac{(\vb{p}+\vb{P})^{2} + r}{2} 
		\notag \\ &\times\bigg[
			\Phi_{\mu}(\vb{p}+\vb{P},t) \delta(\vb{k}-\vb{p}) \delta_{\vb{G},\vb{P}}
			\notag \\& 
			+
			\Phi_{\mu}(-\vb{p}-\vb{P},t) \delta(\vb{k}+\vb{p}) \delta_{-\vb{G},\vb{P}}
		\bigg]
	\notag \\
	\Leftrightarrow\ \comm{\pi_{\mu}(\vb{k}+\vb{G},t)}{\mt{H}_{\Phi}} &= 
		i \Big( (\vb{k}+\vb{G})^{2} + r \Big) \Phi_{\mu}(\vb{k}+\vb{G},t)
\end{align}
%
Inserting both commutator resuls in equation \eqref{eq: second item of equation chain} yields
%
\begin{align}
	\omega \green{\pi_{\mu}(\vb{k}+\vb{G},t)}{\Phi_{\mu'}(\vb{k}'+\vb{G}',t')}_{\omega} = 
		-i (2\pi)^{2} \delta(\vb{k}'+\vb{k}) \delta_{-\vb{G}',\vb{G}} \delta_{\mu,\lambda} \delta_{t,t'} 
		\notag \\  \hspace{2cm} + 
		i\Big( (\vb{k}+\vb{G})^{2} + r \Big) \green{\Phi_{\mu}(\vb{k}+\vb{G},t)}{\Phi_{\mu'}(\vb{k}'+\vb{G}',t')}_{\omega}
\end{align}
%
where on the right hand side the initial wanted Green function appears.
Puting all together we get an algebraic equation for the spin density wave propagator.
Solving this equation yields
%
\begin{align}
	\green{\Phi_{\mu}(\vb{k}+\vb{G},t)}{\Phi_{\lambda}(\vb{k}'+\vb{G}',t')}_{\omega} = (2\pi)^{2} \delta(\vb{k}'+\vb{k}) \delta_{-\vb{G}',\vb{G}} \delta_{\mu,\lambda} \mathcal{D}_{\mu\lambda}^{(0)}(\vb{k},\omega)
\end{align}
%
where for resons of simplicity the free spin density propagator $\mathcal{D}^{(0)}$ is introduced.
%
\begin{align}
	\mathcal{D}_{\mu\lambda}^{(0)}(\vb{k},\omega) := \sum\limits_{\vb{G}} \frac{1}{(\vb{k}+\vb{G})^{2} + r - \epsilon \omega^{2}}
\end{align}
%
In computation of the conductivity below the free Propagator isn't only needed.
The damping of the spin density waves has to be taken into account in the considered model, which is ususally done via pertubation theory.
The pertubation of the spin density waves is caused by the interaction with the electrons living on differant Fermi suffaces.
Therefore the free electron propagator is needed.
In the same way as the free spin density propagator is calculated the free electron propagator can be calculated.
This handwork isn't done here explicitly.
The result should be enough for the moment, which is
%
\begin{align}
	 \green{\Psi_{\alpha}(\vb{k}+\vb{G},t)}{\Psi_{\beta}^{\dag}(\vb{k}'+\vb{G}',t')}_{\omega} = (2\pi)^{2} \delta(\vb{k}'-\vb{k}) \delta_{\vb{G}',\vb{G}} \delta_{\alpha,\beta} \mathcal{G}_{\alpha\beta}^{(0)}(\vb{k},\omega)
\end{align}
%
with
%
\begin{align}
	\mathcal{G}_{\alpha\beta}^{(0)}(\vb{k},\omega) = \sum\limits_{\vb{G}} \frac{1}{\omega - \epsilon_{\alpha}(\vb{k}+\vb{G})},
\end{align}
%
where $\alpha, \beta = \mt{a,b}$ and denotes the typ of electron with respect to the Fermi surfaces.
%
%
\section{The damped spin density wave propagator}
\label{sec: damped propagator}
%
%
The full Green function of spin density waves is given by
%
\begin{align}
	\mathcal{D}_{\mu\lambda}(\vb{k}, \omega) = -i \frac{\expval{\mathcal{T}_{t} \mt{U}(\infty, -\infty) \Phi_{\mu}(\vb{k}+\vb{G},t) \Phi_{\lambda}(\vb{k}'+\vb{G}',t')}_{0}}{\expval{\mathcal{T}_{t} \mt{U}(\infty, -\infty)}_{0}}
\end{align}
%
where the index $0$ means that the expectation value is taken with respect to the unpertubated Hamiltonian.
Thus the whole pertubation is only going into account throught the time evolution operator $\mt{U}$, defined by
%
\begin{align}
	\mt{U}(t,t') = \exp\bigg(-i\int_{t'}^{t} \dd{t_{1}} \mt{H}_{\mt{int}}(t_{1})\bigg)
\end{align}
%
The time evolution operator is expanded into a series up to the second order.
Like it is known from lectures about quantum field theory the denominator denotes the vacuum energy and can be canceled out with terms of the numerator.
In quantum field theory language this means that only connected diagrams have to be investigated.
%
\begin{align}
	\mathcal{D}_{\mu\lambda}^{(1)}(\vb{k}, \omega) &= 
		-i \expval{\mathcal{T}_{t} \Phi_{\mu}(\vb{k}+\vb{G},t) \Phi_{\lambda}(\vb{k}'+\vb{G}',t')}_{0}
		\notag \\&+
		(-i)^{2} \int\limits_{-\infty}^{\infty} \dd{t_{1}} \expval{\mathcal{T}_{t} \mt{H}_{\Phi}(t_{1}) \Phi_{\mu}(\vb{k}+\vb{G},t) \Phi_{\lambda}(\vb{k}'+\vb{G}',t')}_{0}
		\notag \\&+
		(-i)^{3} \int\limits_{-\infty}^{\infty} \dd{t_{1}} \dd{t_{2}} \expval{\mathcal{T}_{t} \mt{H}_{\Phi}(t_{2}) \mt{H}_{\Phi}(t_{1}) \Phi_{\mu}(\vb{k}+\vb{G},t) \Phi_{\lambda}(\vb{k}'+\vb{G}',t')}_{0}
\end{align}
%
The expectation value in the first row correspondes to the free propagator.
The term in the second row vanishes, which is easily seen by regarding the Hamiltonian.
At first the expectation value of the fermionic and bosonic operators can be seperated.
Doing this an expectation value with three bosonic operators appear, which is zero.
The term in the last row is the first non-disapearing contribution.
Let us investigate this beast at first seperatly.
%
\begin{align}
	\mt{E}^{(2)} :&= 
		(-i)^{3} \lambda^{2} \int\limits_{-\infty}^{\infty} \dd{t_{1}} \dd{t_{2}} 
		\sum\limits_{\vb{P}_{1} \vb{P}_{2}} \int_{\vb{p}_{1}} \int_{\vb{p}_{2}}
		\sum\limits_{\vb{P}_{3} \vb{P}_{4}} \int_{\vb{p}_{3}} \int_{\vb{p}_{4}}
		\notag \\ &\times
		\expval{
			\mathcal{T}_{t} 
			\Phi_{\nu'} (\vb{p}_{3}-\vb{p}_{4}+\vb{P}_{3}-\vb{P}_{4},t_{2}) 
			\Phi_{\nu} (\vb{p}_{1}-\vb{p}_{2}+\vb{P}_{1}-\vb{P}_{2},t_{1}) 
			\Phi_{\mu}(\vb{k}+\vb{G},t) 
			\Phi_{\lambda}(\vb{k}'+\vb{G}',t')
		}_{0}
		\notag \\ &\times \bigg(
		\expval{
			\mathcal{T}_{t} 
			\Psi_{\mt{a}}^{\dag}(\vb{p}_{3}+\vb{P}_{3},t_{2}) 
			\cdot \sigma_{\nu'} \cdot 
			\Psi_{\mt{b}}(\vb{p}_{4}+\vb{P}_{4},t_{2})
			\Psi_{\mt{a}}^{\dag}(\vb{p}_{1}+\vb{P}_{1},t_{1}) 
			\cdot \sigma_{\nu'} \cdot 
			\Psi_{\mt{b}}(\vb{p}_{2}+\vb{P}_{2},t_{1})
		}_{0}
		\notag \\ &+
		\expval{
			\mathcal{T}_{t}
			\Psi_{\mt{a}}^{\dag}(\vb{p}_{3}+\vb{P}_{3},t_{2}) 
			\cdot \sigma_{\nu'} \cdot 
			\Psi_{\mt{b}}(\vb{p}_{4}+\vb{P}_{4},t_{2})
			\Psi_{\mt{b}}^{\dag}(\vb{p}_{1}+\vb{P}_{1},t_{1}) 
			\cdot \sigma_{\nu'} \cdot 
			\Psi_{\mt{a}}(\vb{p}_{2}+\vb{P}_{2},t_{1})
		}_{0}
		\notag \\ &+
		\expval{
			\mathcal{T}_{t}
			\Psi_{\mt{b}}^{\dag}(\vb{p}_{3}+\vb{P}_{3},t_{2}) 
			\cdot \sigma_{\nu'} \cdot 
			\Psi_{\mt{a}}(\vb{p}_{4}+\vb{P}_{4},t_{2})
			\Psi_{\mt{a}}^{\dag}(\vb{p}_{1}+\vb{P}_{1},t_{1}) 
			\cdot \sigma_{\nu'} \cdot 
			\Psi_{\mt{b}}(\vb{p}_{2}+\vb{P}_{2},t_{1})
		}_{0}
		\notag \\ &+
		\expval{
			\mathcal{T}_{t} 
			\Psi_{\mt{b}}^{\dag}(\vb{p}_{3}+\vb{P}_{3},t_{2}) 
			\cdot \sigma_{\nu'} \cdot 
			\Psi_{\mt{a}}(\vb{p}_{4}+\vb{P}_{4},t_{2})
			\Psi_{\mt{b}}^{\dag}(\vb{p}_{1}+\vb{P}_{1},t_{1}) 
			\cdot \sigma_{\nu'} \cdot 
			\Psi_{\mt{a}}(\vb{p}_{2}+\vb{P}_{2},t_{1})
		}_{0}
		\bigg)
\end{align}
%














