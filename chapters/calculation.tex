\chapter{Calculation}
\label{ch: calculation}

In the last chapter the memory-matrix-formalism was introduced, which give us an exact formula to calculate correlation functions.
Now this formalism is used to determine the static conductivity of the spin-fermion-model, see chapter (\todo{make link to chapter spin-fermion-model}), pertubated by umklapp-scattering.


\section{Infinite conductivity in systems with unbroken translation symmetry}
\label{sec: Infinite conductivity in a system with unbroken translation symmetry}
%
%
After Drude published his theory about the electrical transport in metals \cite{Drude} in the beginning of the last century it is well known that a broken translation symmetry is needed to get a finite static conductivity.
Because of Neother's theorem it is also well known that a unbroken symmetry always implies a conserved quantity.
In the case of translation symmetry this quantity is the momentum.
Phenomenas breaking the translation symmetry are for example impurity scattering, electron-electron scattering and umklapp scattering.
Let us firstly investigate the standard spin-fermion-model without a translation symmetry breaking pertubation.
In chapter \ref{ch: spin fermion model} it is showed that the unpertubated Hamiltonian conserves the momentum but dosen't conserves the current.
This property is utilized to calculate the static conductivity.

In general the static conductivity is given by taking the small frequency limit of the conductivity and the conductivity itself is given by the current-current correlation function (J-J correlation function). This can be proven by assuming a oscillating electrical field and compute the expactaion value of the current via linear response theory, which is done in \cite{Chycholl2}.
%
\begin{align}
	\sigma_{\mt{dc}} = \lim\limits_{\omega \to 0} \sigma(\omega) = \lim\limits_{\omega \to 0}\, \beta\,\mathcal{C}_{\mt{JJ}}(\omega)
	\label{eq: general static condictivity}
\end{align}
%
In chapter \ref{ch: memory matrix formalism} above the memory matrix formalism is introduced. 
Our main goal was to establish equation \eqref{eq: algebraic equation for C} which is an algebraic matrix equation for the correlation function.
Before the computation of $\mathcal{C}_{\mt{JJ}}(\omega)$ can be started we have to clarify the set of operators over which we sum up.
The sums over $k$ and $l$ arise from the projection operator which means we have to discuss the Liouville subspace into the projection operator projects.
In general to choice of these operators has to be done for each calculation seperatly depending on the working model and the quantity of interest.
In this case the electrical conductivity and the induction of umklapp scattering at its is computated.
As it is said above the electrical conductivity is proportional to the current operator, why this should be the first operator of our sought set of operators.
If an electrical field is applied the electrons accelareate because of the potential difference which increase the momentum of the electrons.
Thus the momentum is an inevitable quantity speaking about current and electrical conductivity this should be the second operator.
Beside these two operators now more operators are necessary.

The current and momentum have the same signature with respect to time reversal symmetry which simplifies the computation a lot.
Considering a invariant Hamiltonian under time reversal symmetrie.
Than in equation \eqref{eq: algebraic equation for C} $\Omega_{il}$ vanishes if both operators have the same signature under time reversal symmetry.
This assertion is proven in section \ref{subsec: time reversal symmetry} in detail.
In addition let do the investigation of $\Sigma_{il}$.
The expactation value is generated with respect to the derivative of an operator at each side. \todo{bessere Formulierung finden}
On the right hand side the sum over $k$ has to be carried out which produces $\toket{\dot{\mt{P}}}$ and $\toket{\dot{\mt{J}}}$.
The first one is trivially zero, because the momentum is a conserved quantity.
The latter has to be investigated under the action of the operator $\mt{Q}$, which projected out off the J-P-subspace.
$\mt{Q}\toket{\dot{\mt{J}}}$ describes the coupling on all the outher degrees of freedom in the system which is zero in the considered system.\todo{besser formulieren}
With all these simplifications equation \eqref{eq: algebraic equation for C} yields
%
\begin{align}
	\begin{pmatrix}
	\mathcal{C}_{\mt{JJ}}(\omega) &  \mathcal{C}_{\mt{JP}}(\omega) \\
	\mathcal{C}_{\mt{PJ}}(\omega) &  \mathcal{C}_{\mt{PP}}(\omega)
	\end{pmatrix}
	=
	\frac{i}{\beta}
	\begin{pmatrix}
	\omega^{-1} & 0 \\
	0 & \omega^{-1} 
	\end{pmatrix}
	\cdot
	\begin{pmatrix}
	\chi_{\mt{JJ}}(\omega) &  \chi_{\mt{JP}}(\omega) \\
	\chi_{\mt{PJ}}(\omega) &  \chi_{\mt{PP}}(\omega)
	\end{pmatrix}
\end{align}
%
where the current current correlation function is given by
%
\begin{align}
	\mathcal{C}_{\mt{JJ}}(z) = \frac{i}{\beta} \omega^{-1} \chi_{\mt{JJ}}(\omega=0) = \frac{i}{\omega} \mathcal{C}_{\mt{JJ}}(t=0),
	\label{eq: correlation function unpertubated system}
\end{align}
%
using relation \eqref{eq: relation between C, Phi and chi}.
The correlation function at $t = 0$ is given by the scalar product $\tobraket{\mt{J}(0)}{\mt{J}(0)}$, see equation \eqref{eq: correlation function Liouville space}.
Nothing or nobody bars us from splitting the vector operator $\toket{\mt{J}(0)}$ into two pieces, one parallel and one vertical part, which corresponds to the secular and non-secular part of the observable, respectivily.
Formaly this look like
%
\begin{align}
	\oket{\mt{J}} = \oket{\mt{J}_{\mid\mid}} + \oket{\mt{J}_{\bot}}.
	\label{eq: splitting current}
\end{align}
%
In general every observable can be consist a conserved and a non-conserved part, what shouldn't mean that both parts exist in every investigated system.
Dissipative prozesses like fluctuations or initial transient processes for example are included in the non-conserved part.
These non-secular effects are visible as noise in the experiement and the vertical part of the vector is indetified with these kinds of prozesses.
Apart from this the secular conserved part of the observable is represented by the parallel part of $\toket{\mt{J}}$.
In Drude's theory of conductivity the current is proportional to the momentum in the way that $j = -\frac{en}{m}p$.
In the spin fermion model, see chapter \ref{ch: spin fermion model}, the momentum is conserved and the current isn't it, which means that the conductivity can't given by Drude's theory at all.
Nevertheless because the momentum is conserved the conserved part of the current has to be in the direction of the momentum.
In mathematical language the parallel part of the current $\toket{\mt{J}_{\mid\mid}}$ is the projection from $\toket{\mt{J}}$ on $\toket{\mt{P}}$.
%
\begin{align}
	\oket{\mt{J}_{\mid\mid}} = \mathcal{P}\oket{\mt{J}} = \frac{\odyad{\mt{P}}{\mt{P}}}{\obraket{\mt{P}}{\mt{P}}} \oket{\mt{J}} = \frac{\chi_{\mt{PJ}}}{\chi_{\mt{PP}}} \oket{\mt{P}}
	\label{eq: parallel current as projection}
\end{align}
%
This give us the oppertunity to write the J-J correlation function into two parts one parrallel and one perpendicular correlation function using equation \eqref{eq: splitting current}.
The mixed correlation functions are zero by construction because $\toket{\mt{J}_{\mid\mid}}$ and $\toket{\mt{J}_{\bot}}$ are orthogonal and therfore the terms vanish.
%
\begin{align}
	\mathcal{C}_{\mt{JJ}}(t=0) = \obraket{\mt{J}(0)}{\mt{J}(0)} = \obraket{\mt{J}_{\mid\mid}}{\mt{J}_{\mid\mid}} + \obraket{\mt{J}_{\bot}}{\mt{J}_{\bot}}
\end{align}
%
Equation \eqref{eq: parallel current as projection} is used to express the parallel J-J correlation function as a momentum-momentum correlation function (P-P correlation) formaly given by $\tobraket{\mt{P}}{\mt{P}}$.
%
\begin{align}
	\mathcal{C}_{\mt{JJ}}(t=0) = \frac{\vert\chi_{\mt{PJ}}\vert^{2}}{\vert\chi_{\mt{PP}}\vert^{2}} \mathcal{C}_{\mt{PP}}(t=0) + \obraket{\mt{J}_{\bot}}{\mt{J}_{\bot}}
\end{align}
%
Using \eqref{eq: relation between C, Phi and chi} and insert back this expression into equation \eqref{eq: correlation function unpertubated system} which give us multipling with $\beta$ the conductivity
%
\begin{align}
	\sigma (z) = \frac{\vert\chi_{\mt{PJ}}\vert^{2}}{\vert\chi_{\mt{PP}}\vert} \frac{i}{\omega}  + \sigma_{\mt{reg}}(\omega)
\end{align}
%
where the regular conductivity $\sigma_{\mt{reg}}(z) = \frac{i \beta}{\omega} \obraket{\mt{J}_{\bot}}{\mt{J}_{\bot}}$ is introduced.
The physical meaning of $\sigma_{\mt{reg}}(\omega)$ is directly connected to the vertical component of $\toket{\mt{J}}$.
Thus the regular conductivity includes fluctuations and other effects influenced by random forces called noise.
Figure \todo{referenz zu bild mit delta peak und rauschen} shows this continuously over all frequencies never disappearing background.

In the whole calculation never a condiction on $\omega$ is made, so the equation for the conductivity is valid for each $\omega$ in the complex plane.
In reality the conductivity isn't depending on a complex frequency, because physical quantities are always real.
Therefore we have to set $\omega = \omega + i \eta$, where now $\omega \in \mathbb{R}$ and the limit $\eta \to 0$ is implied.
Using $\frac{1}{\omega + i\eta} = \mt{PV}\frac{1}{\omega} - i\pi\delta(\omega)$ the conductivity is given by
%
\begin{align}
	\sigma(\omega) = \frac{\vert\chi_{\mt{PJ}}\vert^{2}}{\vert\chi_{\mt{PP}}\vert} \bigg(\mt{PV} \frac{i}{\omega} + \pi \delta(\omega) \bigg) + \sigma_{\mt{reg}}(\omega)
	\label{eq: conductivity unpertubed system}
\end{align}
%
where $\mt{PV}$ sympolizied that the prinzipal value is taken.
Equation \eqref{eq: conductivity unpertubed system} yield us exactly the expected result.
For small frequencies the main contribution is generated by the $\delta$-distribution, so the conductivity becomes infinity.
This isn't really surprising because the translation symmetry isn't broken in the investigated system.
If voltage is applied on a system with unbroken translational symmetry the electrons accelerate infinite long.
There is nothing they can scatter on and loss momentum.
The electrons accelerate more and more and this results in an infinite conductivity.
Only in a system with broken translation symmetry it's possible for the electrons to loss some momentum by scattering with the lattice for example.
This results in a finite conductivity, thus the $\delta$-peak becomes smaller.
The factor in front of the $\delta$-distribution is the so called Drude weight.
The Drude peak and the effect of breaking translation symmetry is visualizied in figure \todo{link to figure} too.
%
%
\section{Finite conductivity because of breaking the translation symmetry via umklapp scattering}
\label{sec: finite conductivity because of breaking the translation symmetry via umklapp scattering}
%
%
The conservation of momentum connected with an unbroken translation symmetry yields a infinite electrical conductivity, which is computated in the section above.
In the next calculation a system with broken translation symmetry is considered.
The assumed symmetry breaking pertubation is umklapp scattering, where the Hamiltonian is given by equation (\todo{link zu umklapp hamiltonian}).
In \dots\todo{link zum abschnitt in dem gezeigt wird das P nicht mehr erhalten ist} it is shown that this pertubation is the reason for an unconserved momentum.
Thus the above disscusion about the Drude weight and conductivity let us expect that the conductivity is lessened to a finite value.
The static electrical conductivity is given by equation \eqref{eq: general static condictivity} in general.
Again the memory matrix formalsim is now used to compute the current-current correlation function given by the formal equation
%
\begin{align}
	\sum\limits_{l} \Big[\omega \delta_{il} - \Omega_{il} + i \Sigma_{il}(\omega)\Big] \mathcal{C}_{lj}(\omega) = \frac{i}{\beta} \chi_{ij}(0)
\end{align}
%
where $\Omega_{il}$ and $\Sigma_{il}(\omega)$ are given by
%
\begin{align}
	&\Omega_{il} = i \beta \sum\limits_{k} \obraket{\dot{\mt{A}}_{i}}{\mt{C}_{k}} \chi_{kl}^{-1}(0) \qq{and} \\
	&\Sigma_{il}(\omega) = i \beta \sum\limits_{k} \obra{\dot{\mt{A}}_{i}} \mt{Q} \frac{1}{\omega - \mt{QLQ}} \mt{Q} \oket{\dot{\mt{C}}_{k}} \chi_{kl}^{-1}(0).
\end{align}
%
Always the first step is to think about the vector subspace, generated by the vectors of the projection operator.
Computing the electrical conductivity the current and the momentum operator are usually the operators of interest. \todo{vllt noch etwas ausf\"uhrlicher schreiben}
Therefore our decision is make and our subspace should be generated by these two operators.
What does this choice of operators mean for the quantities $\Omega_{il}$ and $\Sigma_{il}(\omega)$?
Starting with the first one.
$\Omega_{il}$ vanishs if two properties are valid.
The first one is, that the considered Hamiltonian has to be invariant with respect to time reversal symmetry.
The unpertubated Hamiltonian (\dots\todo{link zum ungest\"orten Hamiltonian}) and the pertubation Hamiltonian (\dots\todo{link zum umklapp Hamiltonian}) occupy this condiction which is trivially to prove.
The second property is that both operators labeled with $\mt{A}_{i}$ and $\mt{C}_{k}$ must have the same signature under time reversal symmetry.
Both operators can be either $\mt{J}$ or $\mt{P}$, where both have the same signature under time reversal symmetry.
Therefore in all cases the quantity $\Omega_{il}$ is zero.
In $\Sigma_{il}(\omega)$ the expecation value is formed with respect of the derivative of vector operators, which are $\toket{\dot{\mt{J}}}$ and $\toket{\dot{\mt{P}}}$.
In the discussion above a translation invariant system is assumed why the derivative of the momentum vanishes.
Now the momentum isn't conserved anymore and the derivative yields a finite value.

For further assertions the action of the operator $\mt{Q}$ on both vector operator has to be investigated.
$\mt{Q} \toket{\dot{\mt{C}}_{k}}$ describes the coupling to all other degrees of freedom which aren't included in the subspace.
Firstly remember that umklapp scattering is the considered pertubation.
What does this pertubation change in our system?
It breaks translation symmetry which yields some finite value for $\dot{\mt{P}}$ instead of zero in the unpertubated system.
This means the complete unconserved part of the momentum is coupled to the crystal lattice which is clearly a degree of freedom out off the J-P subspace.
This is the reason why $\mt{Q} \toket{\dot{\mt{P}}} = \toket{\dot{\mt{P}}}$.
Further the pertubation doesn't change the quantity $\toket{\dot{\mt{J}}}$.
The unconserved current yields from the interaction between the electrons lives on differant Fermi spaces coupeld via spin density waves.
This process is included in the J-P subspace and therefore $\mt{Q} \toket{\dot{\mt{J}}} = 0$.
This signifies for the memory function that $\Sigma_{il}$ doesn't vanish if $i=\mt{P}$ and vanish if $i=\mt{J}$.

In summary umklapp scattering yields a non-zero contribution to the memory function $\Sigma_{il}(\omega)$ and is therefore a correction of the correlation function instead of the unpertubated case where the memory function is zero.
Equation \eqref{eq: algebraic equation for C} yields 4 equations in the J-P subspace, which can be writen as a matrix equation.
%
\begin{align}
	\begin{pmatrix}
	\omega & 0 \\
	-i\Sigma_{\mt{PJ}}(\omega) & \omega - i\Sigma_{\mt{PP}}(\omega)
	\end{pmatrix}
	\cdot
	\begin{pmatrix}
	\mathcal{C}_{\mt{JJ}}(\omega) &  \mathcal{C}_{\mt{JP}}(\omega) \\
	\mathcal{C}_{\mt{PJ}}(\omega) &  \mathcal{C}_{\mt{PP}}(\omega)
	\end{pmatrix}
	=
	\frac{i}{\beta}
	\begin{pmatrix}
	\chi_{\mt{JJ}}(0) &  \chi_{\mt{JP}}(0) \\
	\chi_{\mt{PJ}}(0) &  \chi_{\mt{PP}}(0)
	\end{pmatrix}
	\label{eq: matric equation correlation function unconserved momentum}
\end{align}
%
Before the computation is going on we want to make a short remark.
Equation \eqref{eq: algebraic equation for C} is an exact algebraic matrix equation.
At the derivation no assumtions are made and up to this point we have also made no assumptions.
All the conversion we have done are exact and only depending on the considered model.

The electrical conductivity is given by the J-J correlation function, which has the formal expression
%
\begin{align}
	\mathcal{C}_{\mt{JJ}}(\omega) = \obra{\mt{J}} \frac{i}{\omega - \mt{L}} \oket{\mt{J}}
\end{align}
%
in frequenzy space.
Equally to the case of conserved momentum nothing bars us to split the current into one parallel and one vertical part, where the parallel part is pointed in the direction of the secular component of J.
The appearing mixed correlation functions vanishes because $\toket{\mt{J}_{\mid\mid}}$ and $\toket{\mt{J}_{\bot}}$ are orthogonal.
How we have seen in the previous section the background or noise originated by fluctuation and other random processes is represented by the correlation function of the vertical component.
This term isn't necessary to write it every time down.
A theoretical phyisicist would say that the origin is always taken arbitrary.
A experimental phyisicist would say that he calibrates the measurement.
For a discussion in more detail the work of Jung \cite{Jung} is suggested.
However the only important part for us is the parallel component of the correlation function.
On the other hand the parallel componend of the correlation function is given by the projection of J onto P, see equation \eqref{eq: parallel current as projection}.
Thus the J-J correlation function is rewriten in a momentum -momentum correlation function mutiplied with a fraction of some susceptibilities.
%
\begin{align}
	\mathcal{C}_{\mt{JJ}}(\omega) = \obra{\mt{J}_{\mid\mid}} \frac{i}{\omega - \mt{L}} \oket{\mt{J}_{\mid\mid}} = \frac{\vert\chi_{\mt{PJ}}\vert^{2}}{\vert\chi_{\mt{PP}}\vert^{2}} \mathcal{C}_{\mt{PP}}(\omega)
\end{align}
%
The P-P correlation function can be readed out of equation \eqref{eq: matric equation correlation function unconserved momentum}.
Therefore the invers of the memory matrix has to be multiplied from the left hand side.
The P-P correlation function is given by
%
\begin{align}
	\mathcal{C}_{\mt{PP}}(\omega) = \frac{i}{\beta} \cdot \frac{i \Sigma_{\mt{PJ}}(\omega)  \chi_{\mt{JP}}(0)}{\omega\big(\omega - i\Sigma_{\mt{PP}}(\omega)\big)} + \frac{i}{\beta} \cdot \frac{\chi_{\mt{PP}}(0)}{\omega - i\Sigma_{\mt{PP}}(\omega)} \approx \frac{i}{\beta} \cdot \frac{i \chi_{\mt{PP}}(0)}{\Sigma_{\mt{PP}}(\omega)}
\end{align}
%
where in the last step the limit of small frequencies is taken.
Then on the one hand the first term is neglectable compared to the second term. \todo{Warum ist der erste Term vernachl\"assigbar. Begr\"undung?}
On the other hand is $\omega \ll \Sigma_{\mt{PP}}(\omega)$.
Thus in the second term $\omega$ is neglectable against $\Sigma_{\mt{PP}}(\omega)$.
In summary the static conductivity is given by
%
\begin{align}
	\sigma_{\mt{dc}} = \lim\limits_{\omega \to 0} \beta \mathcal{C}_{\mt{JJ}}(\omega) = \frac{i}{\beta} \lim\limits_{\omega \to 0} \frac{\vert\chi_{\mt{PJ}}\vert^{2}}{\chi_{\mt{PP}}} \frac{i \beta}{\Sigma_{\mt{PP}}(\omega)}
\end{align}
%
The memory function $\Sigma_{\mt{PP}}(\omega)$ is definied in equation \eqref{eq: Sigma(z)}.
Because of the considered Hamiltonian only the term included $\dot{P}$ yields a non-zero contribution.
Further the operator $\mt{QLQ}$ can be approximated by $\mt{L}_{0}$ the Liouville operator of the unpertubated system. \todo{Warum darf QLQ mit $L_{0}$ approximiert werden?}
The final expression for the dc-conductivity is given by
%
\begin{align}
	\sigma_{\mt{dc}} \approx \frac{i}{\beta} \lim\limits_{\omega \to 0} \vert\chi_{\mt{PJ}}\vert^{2} \obra{\dot{\mt{P}}} \frac{1}{\omega - \mt{L_{0}}} \oket{\dot{\mt{P}}}^{-1}
\end{align}
%
In a short conversion the expectation value can be expressed as a time integral over the $\dot{\mt{P}}$-$\dot{\mt{P}}$ susceptibility.
This expression is more usefull for explicite computations, because its allow us to use the Matsubara formalism.
For the detailed conversion see appendix \ref{app: conversion expval}.
%
\begin{align}
	\sigma_{\mt{dc}} \approx -\hbar \lim\limits_{\omega \to 0} \frac{\omega \vert \chi_{\mt{JP}}(\omega = 0) \vert^{2}}{\int\limits_{0}^{\infty} \dd{t} e^{i\omega t} \expval{\comm{\dot{\mt{P}}(t)}{\dot{\mt{P}}(0)}}_{0}}
\end{align}
%
This formula of the static conductivity is the final expression which is used in the compuation below.
The calculation is splitted into two parts.
At first the computation of the denominator is perfermed, which gives us the temperature dependence of the conductivity.
Further the J-P susceptibility has to be calculated.
In first order form this quantity no temperature dependence is expected, but we have to convience us from this.
%
%
\subsection{Temperature dependence of the dc-conductivity}
\label{subsec: temperature dependence of the dc-conductivity}
%
%
Our starting point is the integral in the denominator of the last equation above.
The index $0$ at the expectation value means that it has to be computed with respect to the Hamiltonian $\mt{H}_{1} = \mt{H}_{\Psi} + \mt{H}_{\Phi} + \mt{H}_{\Psi\Phi}$.
The considered umklapp scattering is only entered in the time derivative of the momentum.
Commonly the sort of this calculation is done in the Matsubara time $\tau = it$, see e.\,g. \cite{Bruus&Flensberg} for an introduction or a review.
%
\begin{align}
	\mt{I}(z) :=  -\int\limits_{0}^{\infty} \dd{t} e^{iz t} \expval{\comm{\dot{\mt{P}}(t)}{\dot{\mt{P}}(0)}}_{\mt{H}_{1}} = i \int\limits_{0}^{\beta} \dd{\tau} e^{z \tau} \expval{\comm{\dot{\mt{P}}(\tau)}{\dot{\mt{P}}(0)}}_{\mt{H}_{1}}
\end{align}
%
To symbolisied clearly that the frequency is an arbitrary complex number the variable $z$ is used instead $\omega$ at this point.
Like it is done every time in pertubation theory the operators are transformed into the Matsubara interaction representation.
The transformation's aim is that the expectation value is only taken with respect to the free Hamiltonians $\mt{H}_{0} = \mt{H}_{\Psi} + \mt{H}_{\Phi}$ and the interation $\mt{H}_{\Psi\Phi}$ is only entered in the time evolution operator $\mt{U}(\beta,0)$.
A series expansion of the time evolution operator up to the first non-disappearing order yields
%
\begin{align}
	\mt{I}(z) = i \int\limits_{0}^{\beta} \dd{\tau} e^{z \tau} \expval{\comm{\dot{\mt{P}}(\tau)}{\dot{\mt{P}}(0)}}_{\mt{H}_{0}}
\end{align}
%
In chapter \ref{ch: spin fermion model} umklapp scattering is introduced as a pertubation of the spin fermion system described by $\mt{H}_{1}$.
On the basis of this pertubation the momentum isn't anymore conserved, thus the time derivative of the momentum doesn't vanish.
The time derivative of an operator is given via the Heisenberg equation of motion, which yields for the momentum
%
\begin{align}
	\dot{\mt{P}}_{x} = \frac{i}{\hbar} \sum\limits_{\vb{K}} \mt{J}_{\vb{K}} \int_{\vb{k}} K_{x} \Phi_{\mu}(\vb{k},\tau) \Phi_{\mu}(-\vb{k} - \vb{K},\tau)
\end{align}
%
where the direction of the momentum is set to the $x$-direction without loss of generality.





























