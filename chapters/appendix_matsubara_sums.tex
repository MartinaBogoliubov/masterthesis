\chapter{Analysis of Matsubara-sums}
\label{app: analysis of Matsubara-sums}

In the following appendix it is shown how to calculate two kinds of Matsubara-sums, where the diffrence is depending on the kind of singularity of thee Green-functions.
The first one have simple poles so that the sum can transform without any problems into a contour integral.
These Matsubara-sums are easy to calculate by using the residue theorem.
The second kind of sum contains one or more Green-functions, which have non-continuity at a arbitary value.
Therefore a little bit more work is to do, nevertheless the calculation isn't very complicated.
These type of singularities are called branch cuts.

\section{Simple poles} 
\label{app: simples poles}
Let us assume a Matsubara-sum like
%
\begin{align}
	S(i\omega_{n}) := \frac{1}{\beta} \sum\limits_{\omega_{n}} G(k,i\omega_{n}) e^{i\omega_{n}\tau},
\end{align}
%
where $G(k,i\omega_{n})$ is a product of Green-functions, which are analytical except single poles in the complex plane.
Often these kinds of sums appear by using Green-functions of free propagators.
The exponential function is only needed for conergent.