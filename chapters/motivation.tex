%
%
\chapter{Introduction}
\label{ch:introduction}
%
%
The electrical conductivity or resistance is a fundamental characteristic and an accessable observable of metals.
The response of electrons as a consequence of an external applied electrical field is measured by the conductivity and its magnitude is finite due to internal processes.
A first physical explanation describing this circumstances was published by Drude \cite{Drude} in the year 1900.
He supposed the electrons to be free particles moving around in the metal.
As a consequence of scattering, the electrons are assumed to possess a finite relaxation time $\tau$.
The passing time between to scattering events is denoted by this characteristical variable.
At low temperatures, the electrical conductivity is determined by scattering between electrons and impurities.
The fact, that non-conservation of momentum due to relaxation and a finite electrical conductivity are directly associated, is demonstrated by Drude's theory.
The results of this phenomenological theory are exactly confirmed by using the technique of quantum field theory without considering quantum correction to describe the electrons \cite{Bruus&Flensberg}.
Landau's Fermi liquid theory is the well proven quantum mechanical describtion for metals.
%Metals are well described by Landau's Fermi liquid theory in quantum mechanics.
A quadratic temperature dependence of the resistance $\rho(T) = \rho_{0} + A \cdot T^{2}$, where $\rho_{0}$ is the saturation resistance from impurity scattering at low temperatures \cite{Bader,Pal} is generated by taking into account electron-electron interaction as well as umklapp scattering.
This temperature dependence is modified by quantum mechanical corrections, like weak localization \cite{Altshuler} and the Kondo effect \cite{Kondo}.
A logarithmical increase of the resistance at small temperatures $T < T_{\mt{K}} \approx 1-10\,\mt{K}$ \cite{Kouwenhoven&Glazman} due to electron scattering with localizied magnetic impurities is described by the Kondo effect.
These modifications in the temperature dependence of the resistance are delineated well by Landau's Fermi liquid theory considering quantum corrections.
A non-Fermi liquid behaviour, that is characterizied in a linear temperature dependence of the resistance $\rho(T) = \rho_{0} + A \cdot T$ \cite{Loehneysen}, is  discovered in alloys, like $\mt{CeCu}_{5.9}\mt{Au}_{0.1}$ and $\mt{CeCu}_{5.8}\mt{Ag}_{0.2}$.

The origin of this strange and unexpected behaviour is a quantum phase transition of second order.
In comparison to normal phase transitions, which result from thermal fluctuations, quantum phase transitions are generated by quantum fluctuations.
The phase diagram of such a quantum phase transition is characterizied by a quantum critical point at temperature $T=0$ and a critical tuning parameter $\mt{g}=\mt{g}_{\mt{c}}$.
This tuning parameter has to be a physical parameter, like external pressure, an external applied field or the chemical composition.
It is not possible to reach the absolute zero, but this does not mean that the existence of the quantum critical point and the corresponding quantum fluctuations can not be measured.
At temperatures of several mK, the magnitude of quantum fluctuations is comparable to thermal fluctuations.
The interplay between fermionic Landau quasiparticles and bosonic fluctuations constitute the physical origin of these quantum fluctuations.
The behaviour of quasiparticles becomes critical and as a conseqence of that they acting back to the bosonic fluctuations \cite{Abrahams&Schmalian&Woelfle}
A strong coupling carried by bosonic fluctuations emerges between fermions.

An antiferromagnetic quantum phase transition is generated in the heavy-fermion compounds $\mt{CeCu}_{5.9}\mt{Au}_{0.1}$ and $\mt{CeCu}_{5.8}\mt{Ag}_{0.2}$.
A microscopical description is presented by Abanov et.\,al. \cite{Abanov&Chubukov&Schmalian} in the two dimensional spin-fermion-model containing fermionic quasiparticles and spin density waves.
Spin fluctuations are created at the vicinity of the quantum critical point as a result of permanent particle-holes excitations.
The propagation of these fluctuations is described by the dynamical magnetic susceptibility.
A sharp peak at a large momentum vector $\vb{Q}$ is visable in its spectrum.
As a consequence, a strong interaction between fermions on the Fermi surface seperated by this vector is caused by spin fluctuations.
These certain points on the Fermi surface are called hot spots.

Patel and Sachdev \cite{Patel&Sachdev} make use of the spin-fermion-model modified with an \linebreak anisotropic parapolical fermionic dispersion to discribe the linear temperature dependence of the resistance.
Their computation is based on the memory-matrix-formalism that was firstly pupblished by Mori \cite{Mori} in the year 1965.
The memory-matrix-formalism is grounded on two fundamental ideas: 
1) The ensemble density is seperated into two parts.
One of them contains all the \emph{relevant} amounts determining the mean value of specific observables, while the remaining amounts are denoted as \emph{irrelevant} \cite{Zwanzig}.
2) The memory function $\mt{M}(z)$ decays exponentially in time, $\mt{M}(z) \sim \exp(-\flatfrac{t}{\tau_{\mt{c}}})$, where $\tau_{\mt{c}}$ is the correlation time.
The time evolution of an autocorrelation function is determined by the memory function containing the history of the correlation function \cite{Berne&Boob&Rice}.
In general, the time evolution is described by the memory-matrix-formalism for small time scales.
The memory function decays very slow to zero in the case of conserved quantities.
The time evolution is consequently detemined for large times or small frequencies.
This circumstance is used by Patel and Sachdev to compute the static electrical conductivity in the spin-fermion-model perturbed by a random potential term for fermions and a random-mass term for the bosonic field \cite{Patel&Sachdev}.
A linear resistance is shown using this treatment in both cases.

Beside these two effect, umklapp scattering has also to be taken into account as a possible origin for the linear temperature behaviour of the resistance.
The calculation of the static electrical conductivity similar to the treatment of Patel and Sachdev is presented in this thesis.

\paragraph{Outline of this thesis}$\:$\vspace{3.5ex}

In chapter \ref{ch:infinite conductivity} it is shown that the static electrical conductivity is generally infinite in a system that conserve momentum and do not conserve current. 
A short review over the memory-matrix-formalism is given.
Two fundamental conditions are therefore required: 
1) The Hamiltonian is invariant with respect to time reversal symmetry.
2) The corresponing variables are momentum and current.
Both quantities are assumed to have the same sign under time reversal symmetry.

The spin-fermion-model is introduced in chapter \ref{ch:spin fermion model}.
It is used in our later calculation in chapter \ref{ch:calculation}.
An overview over the principles of quantum phase transitions and appearing characteristic parameters is given.
The temperature dependence of the correlation length is discussed.
The spin-fermion-model is discussed in detail: the propagator of spin fluctuations and the appearance of damping are explained.
Furthermore, the basic concepts of the hot spots theory is presented.
The conservation of momentum and current is illustarted for the spin-fermion model in two cases:
1) unperturbed and 
2) perturbed by umklapp scattering.

In chapter \ref{ch:calculation}, the calculation of the static electrical conductivity is presentet for the spin-fermion-model pertubated by umklapp scattering.
This is divided into two parts.
The temperature dependence of the susceptibility $\chi_{\mt{JP}}$ is determined and the Green function $\mathcal{G}_{\dot{\mt{P}}\dot{\mt{P}}}(\vb{k},z)$ is calculated.
In both cases, the diagrammatic pertubation theory is utilizied, whereas in the latter the integral is analyzied approximately.

Chapter \ref{ch:memory matrix formalism} gives a detailed overview over the memory-matrix-formalism.
The mathematical framework is defined by introducing the Liouville space.
Afterwards, an explicite expression of the autocorrelation function is derivated depending on the memory function $\mt{M}(z)$.
The Evidence that dissipative effect in $\mt{M}(z)$ disappear is proved under the conditions: 
1) The Hamiltonian is invariant with respect to time reversal symmetry.
2) The corresponding operators are assumed to have the same sign under time reversal symmetry.
Finally, the expression to calculate the static conductivity is derived.

\paragraph{Notation}$\:$\vspace{3.5ex}

In this masterthesis, we chose units where $\hbar = k_{\mt{B}} = 1$ and with that $\beta = T^{-1}$.
The intergation over the first two dimensional Brillouin zone and the sums over spatial directions or the fermion species a and b is abbreviated with the symbols:
%
\begin{align}
	&\int_{\vb{k}} := \int\limits_{\mt{BZ}} \frac{\dd[2]{\vb{k}}}{(2\pi)^{2}}
	\\
	&\sum\limits_{\mu} := \sum\limits_{\mu = 1}^{3} \qq{equally} \lambda
	\\
	&\sum\limits_{\alpha} := \sum\limits_{\alpha = a,b} \qq{equally} \beta \qq{and} \gamma
	\\
	&\sum\limits_{\alpha\neq\beta} := \sum\limits_{\substack{\alpha = a,b \\ \alpha\neq\beta}} \sum_{\beta = a,b}
\end{align}
%
Furthermore, these integrals are usefull in our calculation \cite{Bronstein}:
%
\begin{align}
	\int \dd{x} \frac{1}{(z^{2} + 1)^{2}} = \frac{1}{2}\Big[\frac{z}{z^2 + 1} + \tan[-1](z)\Big]
	\label{eq:integral1}
	\\
	\int\limits_{0}^{\infty} \dd{x} \frac{x}{a^{2} + x^{2}} \cdot \frac{x}{y^{2} + x^{2}} = \frac{\pi}{2} \frac{1}{a + |y|}
	\label{eq:integral2}
\end{align}
%




































