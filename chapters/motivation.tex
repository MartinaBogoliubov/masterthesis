%
%
\chapter{Introduction}
\label{ch:introduction}
%
%
The electrical conductivity or resistance is a fundamental characteristic and an accessable observable of metals.
The response of electrons as a consequence of an external applied electrical field is measured by the conductivity and its magnitude is finite due to internal processes.
A first physical explanation describing this circumstances was published by Drude \cite{Drude} in the year 1900.
He supposed the electrons to be free particles moving around in the metal.
As a consequence of scattering, the electrons are assumed to possess a finite relaxation time $\tau$.
This characteristical variable denotes the passing time between to scattering events.
At low temperatures, the electrical conductivity is determined by scattering between electrons and impurities.
Drude's theory demonstrates the fact, that non-conservation of momentum due to relaxation and a finite electrical conductivity are directly associated.
The results of this phenomenological theory are exactly confirmed by using the technique of quantum field theory without considering quantum correction to describe the electrons \cite{Bruus&Flensberg}.
Landau's Fermi liquid theory is the well proven quantum mechanical describtion for metals.
%Metals are well described by Landau's Fermi liquid theory in quantum mechanics.
Electron-electron interaction and the consideration of umklapp scattering generates a quadratic temperature dependence of the resistance $\rho(T) = \rho_{0} + A \cdot T^{2}$, where $\rho_{0}$ is the saturation resistance from impurity scattering at low temperatures \cite{Bader,Pal}.
Quantum mechanical corrections, like weak localization \cite{Altshuler} and the Kondo effect \cite{Kondo}, modify the temperature dependence.
The latter describes a logarithmical increase of the resistance at small temperatures $T < T_{\mt{K}} \approx 1-10\,\mt{K}$ \cite{Kouwenhoven&Glazman} due to electron scattering with localizied magnetic impurities.
Landau's Fermi liquid theory can be described these modifications in the temperature dependence of the resistance.
In contrast, alloys, like $\mt{CeCu}_{6-x}\mt{Au}_{x}$ and $\mt{CeCu}_{6-x}\mt{Ag}_{x}$ at a concentration of $x=0.1$ and $x=0.2$, respectively, exhibit a non-Fermi liquid behaviour, that is characterizied in a linear temperature dependence of the resistance $\rho(T) = \rho_{0} + A \cdot T$ \cite{Loehneysen}. 

The origin of this strange and unexpected behaviour is an antiferromagnetic quantum phase transition of second order.
In comparison to normal phase transiotion, quantum phase transitions are governed by quantum fluctuations instead of thermal fluctuation.
The spin-fermion-model describes those metals at the vinicty of an quantum critical point.


































