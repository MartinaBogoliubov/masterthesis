%
%
%
\chapter{Memory-Matrix-Formalism}
\label{ch:memory matrix formalism} 
%
%
%
The memory-matrix-formalism is a technique to determine autocorrelation functions.
These correlation functions are defined in the Liouville space and are directly connected to their own history by the memory function $\mt{M}(z)$. 
The time evolution of correlation functions decays therefore exponentially in time, $\mt{M}(z) \sim \exp(-\flatfrac{t}{\tau_{\mt{c}}})$, where $\tau_{\mt{c}}$ denotes the correlation or relaxation time.
Considering a conserved quantity, the relaxation time is infinite and the time evolution is consequently determined for large time scales or small frequencies.
The time evolution is only slightly modified, if a pertubation is taken into account in this case.
The correlation function is correctly predicted.

The derivation of the memory-matrix-formalism is exhaustively demonstrated in this chapter, as suggested in \cite{Forster}.
Kubo's relaxation function is firtly introduced, since the correlation function is based on it.
In main part of this chapter a explicit formula for correlation function is derivated.
Finally, this formula is used to find an expression for the static electrical conductivity which is applicable to diagrammatic pertubation theory.
%
%
\section{Kubo Relaxation Function}
\label{sec:kubo relaxation function}
%
%
A system in equlibrium, represented by the Hamiltonian $\mt{H}_{0}$, is considered.
The pertubation $\mt{H}_{1} = -\mt{B} \cdot F(t)$ is switched on at an arbitrary time $t'$.
The coupling to the system is determined by the operator B and the time evolution of the pertubation is described by $F(t)$, where $F(t)$ is assumed to be zero for times $t<t'$.
Our interest is now focused on the reaction of an operator A due to the pertubation.
The deviation in comparing to the equilibrum value is given by
%
\begin{align}
	\delta\expval{\mt{A}(t)} := \expval{\mt{A}}(t) - \expval{\mt{A}(t)}_{H_{0}} \approx \int\limits_{-\infty}^{\infty} \dd{t'} \chi_{\mt{AB}}(t-t') F(t')
	\label{eq:Kubo formula}
\end{align}
%
where $\chi_{\mt{AB}}(t-t')$ is called the retarded susceptibility and is given by
%
\begin{align}
	\chi_{\mt{AB}}(t-t') = i \Theta(t-t') \expval{\comm{\mt{A}_{\mt{I}}(t-t')}{\mt{B}_{\mt{I}}(0)}}_{\mt{H}_{0}}
	\label{eq:dynamical susceptibilty}
\end{align}
% 
The above equation for $\delta\expval{\mt{A}(t)}$ is denoted as the Kubo formula.
In the following, a certain type of pertubations is considered.
The time evolution of the pertubation is assumed to be $F(t) = \Theta(-t) \cdot F \cdot e^{-s\tau}$.
The pertubation is switched on adiabatically at $t = -\infty$ and is switched off at $t=0$.
This time evolution is inserted into the Kubo formula and $t-t'$ is substituted by $\tau$.
The deviation of $\mt{A}(t)$ is then given by $\delta\expval{\mt{A}(t)} = \Phi_{AB}(t) \cdot F e^{st}$.
The arising function $\Phi_{\mt{AB}}(t)$ is called Kubo relaxation function and is given by
%
\begin{align}
	\Phi_{\mt{AB}}(t) = \frac{i}{\hbar} \lim\limits_{s \to 0} \int\limits_{t}^{\infty} \dd{\tau} \expval{\comm{\mt{A}_{\mt{I}}(\tau)}{\mt{B}_{\mt{I}}(0)}}_{0} e^{-s\tau}.
	\label{eq:Kubo relaxation function}
\end{align}
%
The lower limit of the integral is determined to $t$ due to the $\theta$-distribution.
For a more detailed derivation of the Kubo relaxation function see \cite{Schwabl} or \cite{Schwabl2}.
Between the Kubo relaxation function and the Kubo formula exist three very important relations used during the derivation of the correlation function and the formula of the conductivity.
%
%\begin{enumerate}
%	\item $\begin{aligned}[t] \chi_{\mt{AB}}(t) = -\Theta(t) \dv{t} \Phi_{\mt{AB}}(t) \end{aligned}$\hfill \refstepcounter{equation}(\theequation)\label{eq:relation 1 between Phi and chi}
%	\item $\begin{aligned}[t] \Phi_{\mt{AB}}(t = 0) = \chi_{\mt{AB}}(\omega = 0) \end{aligned}$\hfill \refstepcounter{equation}(\theequation)\label{eq:relation 2 between Phi and chi}
%	\item $\begin{aligned}[t] \Phi_{\mt{AB}}(\omega) = \frac{1}{i\omega}\big[\chi_{\mt{AB}}(\omega) - \chi_{\mt{AB}}(\omega = 0)\big]. \end{aligned}$\hfill \refstepcounter{equation}(\theequation)\label{eq:relation 3 between Phi and chi}
%\end{enumerate}
%
The evidence of these tree relations are shown in the appendix \ref{app:properties of the Kubo relaxation function}.
The Kubo relaxation function is now transfered into a commutator independent form, using two identies.
The first identity is given by
%
\begin{align}
	\expval{\comm{\mt{A}(t)}{\mt{B}(t')}} &= \frac{1}{Z} \Tr{\comm{\rho}{\mt{A}(t)} \mt{B}(t')},
	\label{eq:identity expectation value}
\end{align}
%
where the invariance of the trace with respect to cycling permutation is used.
The second identity is denoted as the Kubo-identity.
%
\begin{align}
	i \comm{\rho}{\mt{A}(t)} &= i \Big[\rho \mt{A}(t) - e^{-\beta H} e^{\beta \mt{H}} \mt{A}(t) e^{-\beta \mt{H}}\Big]
	\notag \\
	\Leftrightarrow\ i \comm{\rho}{\mt{A}(t)} &= -i \rho \int\limits_{0}^{\beta} \dd{\lambda} \dv{\lambda} e^{\lambda \mt{H}} \mt{A}(t) e^{-\lambda \mt{H}}
	\notag \\
	\Leftrightarrow\ i \comm{\rho}{\mt{A}(t)} &= -i \rho \int\limits_{0}^{\beta} \dd{\lambda} \comm{\mt{H}}{\mt{A}(t-i\lambda)}
	\notag \\
	\Leftrightarrow\ i \comm{\rho}{\mt{A}(t)} &= -\rho \int\limits_{0}^{\beta} \dd{\lambda} \dot{\mt{A}}(t-i\lambda)
	\label{eq:Kubo-identity}
\end{align}
%
Here the analogy between the time evolution of an operator and the exponential function is used in the second step.
Furthermore, Heisenberg equation of motion is utilied in the third step.
The time derivative of the operator A is symbolizied with the dot above the operator.
%
\begin{align}
	&\Phi_{\mt{AB}}(t) = -\lim\limits_{s \to 0} \int\limits_{0}^{\beta} \dd{\lambda} \int\limits_{t}^{\infty} \dd{\tau} \expval{\dot{\mt{A}}_{\mt{I}}(\tau-i\lambda\hbar) \mt{B}_{\mt{I}}(0)}_{0} e^{-s\tau}
	\notag \\
	\overset{\mt{PI}}{\Leftrightarrow}\ &\Phi_{\mt{AB}}(t) = \int\limits_{0}^{\beta} \dd{\lambda} \expval{\mt{A}_{\mt{I}}(t-i\lambda\hbar) \mt{B}_{\mt{I}}(0)}_{0} = \int\limits_{0}^{\beta} \dd{\lambda} \expval{\mt{A}_{\mt{I}}(t) \mt{B}_{\mt{I}}(i\lambda\hbar)}_{0}
	\label{eq:Kubo relaxation function 2.0}
\end{align}
%
The first line is obtained, inserting both identities.
On the righ hand side the integral is evaluated using integration by parts (PI).
Afterwards, the limit $s\to 0$ is performed, which yields the obatin result for $\Phi_{\mt{AB}}(t)$.
This structure of the Kubo relaxation function is similar to the later chosen scalar product in the memory matrix formalism.

%Later we will see that the scalar product defining at the memory-matrix-formalsim has a similar structure.
%This provide the oppertunity to transform the correlation function out of the language of the memory-matrix-formalism into the Kubo relaxation function, which in turn provide the oppertunity to compute the correlation function pertubativly.

%
%
\subsection{Spectral Representation}
\label{subsec:spectral representation}
%
%
In the previous section, the dynamical susecptibility $\chi_{\mt{AB}}$ is introduced deviating the Kubo-formula \eqref{eq:Kubo formula}.
The evolution of a operator is described by this function, while a pertubation is acting to the system.
The dynamic susceptibility is seperated into to types, non-dissipative and dissipative processes.
In the following, dissipative processes are investigated.
The dissipative susceptibility of the form
%
\begin{align}
	\chi''_{\mt{AB}}(t-t') = \frac{1}{2\hbar} \expval{\comm{\mt{A}(t)}{\mt{B}(t')}}
	\label{eq:dissipative susceptibility}
\end{align}
%
is considered, where the operators $\mt{A}$ and $\mt{B}$ are Hermitian.
The property
%
\begin{align}
	\big(\chi''_{\mt{AB}}(t-t')\big)^{*} = - \chi''_{\mt{AB}}(t-t')
	\label{eq:complex conjugated of dissipative susceptibility}
\end{align}
%
is valid, since the commutator of two Hermitian operators is anti-Hermitian.
In the following, the dynamic susceptibility is expressed using \eqref{eq:dissipative susceptibility}.
The obtained equation is multiplied with $e^{i\omega t}$ and is integrated over time $t$.
%
\begin{align}
	\chi_{\mt{AB}}(t) &= \frac{i}{\hbar} \Theta(t) \expval{\comm{\mt{A}(t)}{\mt{B}(0)}} = 2i \Theta(t) \chi''_{\mt{AB}}(t)
	\notag \\
	\Leftrightarrow\ \chi_{\mt{AB}}(\omega) &= 2i \int\limits_{-\infty}^{\infty} \dd{t} e^{i\omega t} \Theta(t) \chi''_{\mt{AB}}(t)
	\notag \\
	\Leftrightarrow\ \chi_{\mt{AB}}(\omega) &= -\frac{1}{\pi} \lim\limits_{\eta \to 0} \int\limits_{-\infty}^{\infty} \dd{\omega'}
 \frac{1}{\omega' + i\eta} \int\limits_{-\infty}^{\infty} \dd{t} e^{i(\omega-\omega')t} \chi''_{\mt{AB}}(t)
 	\notag \\
	\Leftrightarrow\ \chi_{\mt{AB}}(\omega) &= \frac{1}{\pi} \lim\limits_{\eta \to 0} \int\limits_{-\infty}^{\infty} \dd{\omega'}
 \frac{\chi''_{\mt{AB}}(\omega')}{\omega' - \omega - i\eta} 
 	\notag \\
	\Leftrightarrow\ \chi_{\mt{AB}}(\omega) &= \frac{1}{\pi} \mt{PV} \int\limits_{-\infty}^{\infty} \dd{\omega'}
 \frac{\chi''_{\mt{AB}}(\omega')}{\omega' - \omega} + i \int\limits_{-\infty}^{\infty} \dd{\omega'} \delta(\omega' - \omega) \chi''_{\mt{AB}}(\omega')
 	\notag \\
	\Leftrightarrow\ \chi_{\mt{AB}}(\omega) &= \chi'_{\mt{AB}}(\omega) + i \chi''_{\mt{AB}}(\omega)
	\label{eq:splitting susceptibility into real and imaginary part}
\end{align}
%
On the left hand side the definition of the Fourier transformation is used, while on the right hand side the following definition of the $\Theta$-function is inserted.
%
\begin{align}
	\Theta_{\eta}(t) = i \lim\limits_{\eta \to 0} \int\limits_{-\infty}^{\infty} \frac{\dd{\omega'}}{2\pi} \frac{e^{-i\omega't}}{\omega' + i\eta} 
\end{align}
%
The dynamical susceptibility $\chi_{\mt{AB}}(\omega)$ is seperated into two parts $\chi'_{\mt{AB}}(\omega)$ and $\chi''_{\mt{AB}}(\omega)$, where the latter is the dissipative susceptibility.
Assuming the dissipative susceptibility is a real number, than this is also valid for $\chi'_{\mt{AB}}(\omega)$ and the both functions $\chi'_{\mt{AB}}(\omega)$ and $\chi''_{\mt{AB}}(\omega)$ represent real and imaginary part of $\chi_{\mt{AB}}(\omega)$, respectivily.
After this introduction about the Kubo relaxation function, the next chapter is illustrated the derivation of the memory matrix formalism.
%
%
%
\section{Correlation functions in the Memory-Matrix-Formalism}
\label{sec:deviation of the memory-matrix-formalism}
%
%
%
In order to determine transport properties of many-body-systems, the time evolution of an observable has to be investigated.
The memory-matrix-formalsim is historical introduced to descrie the Brownian motion under the consideration of dissioation and fluctuations.
In \cite{Mori} is demonstrated the seperation of a dynamical variable $\mt{A}(t)$ into two parts, one secular and one non-secular one.
The starting point of this approach is based on the Langevin equation, considering damping of the variable, the coupling to some external force and a random force.
To get a linearizied form of the Lagevin equation, it is assumed to seperate the dynamical variable into a part considering its histery and a part containing all effects of other degrees of freedom.
The first part is now expanded up to the linaear order.
The obtained differential equation is solved using the Laplace transformation.
The dynamical variable is then given by
%
\begin{align}
	\mt{A}(t) = \Xi(t) \cdot \mt{A}(0) + \mt{A}'(t) \qq{with} \mt{A}'(t) = \int\limits_{0}^{t} \dd{t'} \Xi(t-t') F(t').
\end{align}
%
The function $\Xi(t)$ is defined by the Laplace transformation of $\Xi(s) = [s-\mathcal{C}(s)]^{-1}$ and $\mathcal{C}(s)$ is the Laplace transformtion of the correlation function $\mathcal{C}(t)$.
Our further derivation of the memory-matrix-formalism is motivated by this short historical review.
It shows, that a dynamical variable is seperated into to parts.
The first term includes the linear contributions of $\mt{A}(t)$, while the second term contains non-linear effects and fluctuations, for example.
Both terms are identified with secular and non-secular effects, respectivily, while the dynamics of $\mt{A}(t)$ are dominated by the secular ones.

The seperation of $\mt{A}(t)$ offers the opportunity to a simple geometrical interpretation.
The variable $\mt{A}(t)$ is assumed as a vector in a vector space.
The direction of $\mt{A}(t)$ is labeled as the A-axis, if the system is in equilibrium.
If now the system moves out of equilibrium caused by a pertubation, also the direction of $\mt{A}(t)$ changes.
The part parallel to the A-axis is identified with the secular part, while the part perpendicular to the A-axis corresponds to the non-secular part.

A mathematical vector space has to be defined, for using this geometrical interpretation to determine the time evolution of an observable.
Afterwards we define a correlation funtion in this vector space and bring its in much usefuller form.
Our starting point is the usual Hilbert space in quantum mechanics.
A short review based in \cite{Audretsch} is given in the following.

The $d$-dimensional Hilbert-space is linear, complex and has a defined scalar product.
The vectors $\ket{\phi}$, usually denoted in the Dirac-notation, are identified with all possible states of the system.
Since we are always interested in observables, linear Hermitian operators are defined in the Hilbert-space.
The eigenvalues of them conform to observables.
Using the dyad product $\sum_{i} \dyad{i}{i}$, any linear operator can be writen as a dyad decomposition
%
\begin{align}
	\mt{A} = \sum\limits_{i,j} \dyad{i}{i} \mt{A} \dyad{j}{j} = \sum\limits_{i,j} \mt{A}_{ij} \ket{i} \bra{j},
	\label{eq:dyad product}
\end{align}
%
where $\mt{A}_{ij} := \bra{i}\mt{A}\ket{j}$ is the matrix element of the corresponding linear operator.
The dyad product of an operator is now used to introduce a new vector space of all linear operators acting on the $d$-dimensional Hilbert-space, called the Liouville-space $\mathbb{L}$ or operator space.

The Liouville-space is a linear and complex vector space equally to the Hilbert-space.
The difference between both are the vectors or elements living in the space.
In the Liouville-space the vectors are linear operators $\mt{A}, \mt{B}, \dots$ which are acting on a Hilbert-space.
In other words this means that the dyad decomposition of an vector in the $d$-dimensional Hilbert-space is the new vector in the Liouville-space.
A vector in the Liouville-space is notated as
%
\begin{align}
	\oket{\mt{A}} := \sum\limits_{i,j}^{d} \mt{A}_{ij} \oket{\dyad{i}{j}}.
	\label{eq:definition of a vector in Liouville-space}
\end{align}
%
Similiarly to the quantum mechanic the Dirac notation is used with the difference that round brackets are used instead of angle brackets to distinghush both spaces.
Out of the definition \eqref{eq:definition of a vector in Liouville-space} it is clear, that the basis in the Liouville-space is build by the $d^{2}$ dyads of the Hilbert-space .
The dimension of the Liouville-space is therefore $d^{2}$.
Equally to a Hilbert-space, there are many other oppertunities to choose the basis in the Liouville space $\mathbb{L}$, but the defintion in \eqref{eq:definition of a vector in Liouville-space} is the one which we choose in the following.

The basis of our Liouville space is denoted with $\{\toket{\mt{A}_{i}}\}$ where $i = 1,2,3,\dots,n$ and $\mt{A}_{i}$ is an operator.
The corresponding basis of the dual space is given by $\{\tobra{\mt{A}_{i}}\}$, similarily to the Hilbert space.
The last needed element of our Liouville space is a scalar product which has to fulfil the following three condictions.
%
%\begin{enumerate}
	%\item $\begin{aligned} \obraket{\mt{A}_{i}}{\mt{A}_{j}} = \obraket{\mt{A}_{j}}{\mt{A}_{i}}^{*} \end{aligned}$\hfill \refstepcounter{equation}(\theequation)
	%\item $\begin{aligned} \obraket{\mt{A}_{i}}{\mt{B}} = c_{1} \obraket{\mt{A}_{i}}{\mt{A}_{j}} + c_{2} \obraket{\mt{A}_{i}}{\mt{A}_{k}}, \end{aligned}$
	%\item[] $\begin{aligned} \text{where\ \ \ } \mt{B} = c_{1} \mt{A_{j}} + c_{2} \mt{A_{k}} \qq{and} c_{1}, c_{2} \in \mathbb{C} \end{aligned}$\hfill \refstepcounter{equation}(\theequation)
	%\item $\begin{aligned} \obraket{\mt{A}_{i}}{\mt{A}_{i}}\geq 0 \qq{, where equallity is fulfilled if} \mt{A}_{i} = 0. \end{aligned}$\hfill \refstepcounter{equation}(\theequation)
%\end{enumerate}
%
Beside these the choice of the scalar product is arbitrary.
For the moment let us choose 
%
\begin{align}
	\obraket{\mt{A}_{i}(t)}{\mt{A}_{j}(t')} = \frac{1}{\beta} \int\limits_{0}^{\beta} \dd{\lambda} \expval{\mt{A}_{i}^{\dag}(t) \mt{A}_{j}(t'+i\lambda\hbar)}
	\label{eq:scalar product Liouville space2.0}
\end{align}
%
as our scalar product, where the normal time evolution of an operator \linebreak $\mt{A}_{i}(t) = e^{i\mt{H}t/\hbar} \mt{A}_{i}(0) e^{-i\mt{H}t/\hbar}$ is valid, so that $\mt{A}_{i}(i\lambda\hbar) = e^{-\lambda\mt{H}} \mt{A}_{i}(0) e^{\lambda\mt{H}}$ is possible to use.
Now we have to prove, if the condictions are fulfilled by the choice of our scalar product.
The second condiction is shown transforming the expactation value into the trace representation and using the properties of the trace.
%
\begin{align}
	\obraket{\mt{A}_{i}(t)}{\mt{B}(t')} &= \frac{1}{\beta} \int\limits_{0}^{\beta} \dd{\lambda} \frac{1}{Z} \Tr{\rho \mt{A}_{i}^{\dag}(t) \Big[c_{1} \mt{A}_{j}(t'+i\lambda\hbar) + c_{2} \mt{A}_{k}(t'+i\lambda\hbar)\Big]}
	\notag \\
	\Leftrightarrow\ \obraket{\mt{A}_{i}(t)}{\mt{B}(t')} &= c_{1} \obraket{\mt{A}_{i}(t)}{\mt{A}_{j}(t')} + c_{2} \obraket{\mt{A}_{i}(t)}{\mt{A}_{k}(t')}
\end{align}
%
The first and third condition can be shown by transforming the scalar product in the spectral representation.
The trace is writen explicitly as a sum over all states and the unity operator, $1 = \sum_{m} \ket{m} \bra{m}$, is inserted between both operators $\mt{A}_{i}$ and $\mt{A}_{j}$. 
%
\begin{align}
	\obraket{\mt{A}_{i}(t)}{\mt{A}_{j}(t')} &= \frac{1}{\beta \cdot Z} \int\limits_{0}^{\beta} \dd{\lambda} \sum\limits_{n,m} \bra{n} e^{-\beta \mt{H}} \mt{A}_{i}^{\dag}(t) \ket{m} \bra{m} e^{-\lambda \mt{H}} \mt{A}_{j}(t') e^{\lambda \mt{H}} \ket{n}
	\notag \\
	\Leftrightarrow\ \obraket{\mt{A}_{i}(t)}{\mt{A}_{j}(t')} &= \frac{1}{\beta \cdot Z} \sum\limits_{n,m} \mel{n}{\mt{A}_{i}^{\dag}(t)}{m} \mel{m}{\mt{A}_{j}(t')}{n} e^{-\beta E_{n}} \int\limits_{0}^{\beta} \dd{\lambda} e^{\lambda (E_{n}-E_{m})} 
	\notag \\
	\Leftrightarrow\ \obraket{\mt{A}_{i}(t)}{\mt{A}_{j}(t')} &= \frac{1}{\beta \cdot Z} \sum\limits_{n,m} \mel{n}{\mt{A}_{i}^{\dag}(t)}{m} \mel{m}{\mt{A}_{j}(t')}{n}  \frac{e^{-\beta E_{m}} - e^{-\beta E_{n}}}{E_{n}-E_{m}}
	\label{eq:expectation value in spectral representation}
\end{align}
%
The complex conjugated of the expectation value is considered in the Liouville space and the first condiction is instantly proven, using $\mel{n}{\mt{A}_{j}^{\dag}(t)}{m}^{*} = \mel{m}{\mt{A}_{j}(t)}{n}$.
Notice that on the right hand side of equation \eqref{eq:expectation value in spectral representation} only the expactation values are complex quantities.
The complex conjugated of them yields
%
\begin{align}
	\Big(\mel{n}{\mt{A}_{i}^{\dag}(t)}{m} \mel{m}{\mt{A}_{j}(t')}{n}\Big)^{*} = \mel{n}{\mt{A}_{j}^{\dag}(t')}{m} \mel{m}{\mt{A}_{i}(t)}{n}.
\end{align}
%
If this is inserted back in $\tobraket{\mt{A}_{i}(t)}{\mt{A}_{j}(t')}^{*}$, equation \eqref{eq:expectation value in spectral representation} is found..
To prove the third condition $\mt{A}_{j}(t')$ is set to $\mt{A}_{i}(t)$ in equation \eqref{eq:expectation value in spectral representation}
On the right hand side of this equation we obtain
%
\begin{align}
	\frac{1}{\beta \cdot Z} \sum\limits_{n,m} \big\vert\mel{m}{\mt{A}_{i}(t)}{n}\big\vert^{2} \frac{e^{-\beta E_{m}} - e^{-\beta E_{n}}}{E_{n}-E_{m}}.
\end{align}
%
The squared expactation value is always non-negative and the fraction is positive too.
This is proven investigating the two cases $E_{n} > E_{m}$ and $E_{n} < E_{m}$.
Therefore, the expactation value $\obraket{\mt{A}_{i}(t)}{\mt{A}_{i}(t)} \geq 0$ and equality is only possible if $\mt{A}_{i} = 0$.
All three conditions are well proven and the choice of our scalar product is valid.
At this point the definition of our used vector space is complete.
We know the describtion of the vectors and scalar product in the Liouville space

To describe the resluts of measurements, we need a definition of a correlation function.
In the following, a correlation function is derivated in the Liouville space.
The natural starting point to determine the time evolution of an operator $\mt{A}_{i}$ is the Heisenberg equation of motion in quantum mechanics.
%
\begin{align}
	\dv{t} \mt{A}_{i}(t) = \dot{\mt{A}}_{i}(t) = \frac{i}{\hbar} \comm{\mt{H}}{\mt{A}_{i}(t)} = i \mt{L} \mt{A}_{i}(t)
	\label{eq:Heisenberg equation of motion}
\end{align}
%
The operators are in the Heisenberg representation and the Hermitian Liouville operator ${\mt{L} = \hbar^{-1} \comm{\mt{H}}{\mt{\bullet}}}$ is introduced, which is defined by its action on an operator.
The formal solution of equation \eqref{eq:Heisenberg equation of motion} is given by
%
\begin{align}
	\mt{A}_{i}(t) = e^{it\mt{L}} \mt{A}_{i}(0) = e^{it\mt{H}/\hbar} \mt{A}_{i}(0) e^{-it\mt{H}/\hbar}.
\end{align}
%
In the second step, the definition of the Liouville operator and some algebraic transformations are used.
In this notation, it is more clearly that the time evolution of an operator is given by the Liouville operator.
The same result is obtained in the Liouville space if the Liouville operator is acting on the basis vectors.
The following equation is obtained inserting the dyad product in equation \eqref{eq:Heisenberg equation of motion}.
%
\begin{align}
	\oket{\dot{\mt{A}}_{i}(t)} = \frac{i}{\hbar} \oket{\comm{\mt{H}}{\mt{A}_{i}(t)}} = i \mt{L} \oket{\mt{A}_{i}(t)}
	\label{eq:Heisenberg equation of motion in Liouville space}
\end{align}
%
There formal solution is given by
%
\begin{align}
	\oket{\mt{A}_{i}(t)} = e^{it\mt{L}} \oket{\mt{A}_{i}(0)}.
	\label{eq:formal solution of EM in L}
\end{align}
%
Beside the Liouville operator, one more operator, called the projection operator, is introduced for the derivation of the correlation function.
Therefore, we define a set of operators $\{\mt{C}_{i}\}$, where the choice of these operators are differently depending on the investigated system and correlation function.
The choice of these operators is discussed in chapter \ref{ch:infinite conductivity} and later at the derivation of the formula of the static conductivity.
For the moment it is sufficient to know that the set of operators exists.
Directly following from the definition of the projection operator in quantum mechanics the projection operator in the Liouville space possesses the form
%
\begin{align}
	\mt{P} = \sum\limits_{i,j} \oket{\mt{C}_{i}(0)} \obraket{\mt{C}_{i}(0)}{\mt{C}_{j}(0)}^{-1} \obra{\mt{C}_{j}(0)}.
	\label{eq:projection operator2.0}
\end{align}
%
The action of P on a vector $\oket{\mt{A}(t)}$ in the Liouville space yields the parallel components to the chosen operators $\mt{C}_{i}$, which is the projection from $\oket{\mt{A}(t)}$ into the vector subspace spanned by $\mt{C}_{i}$. 
The corresponding vertical component of $\oket{\mt{A}(t)}$ is given by $\mt{Q} = 1- \mt{P}$, which is the projection out of the vector subspace.
Naturally the projection operator is fulfilled the two properties $\mt{P}^{2} = \mt{P}$ and $\mt{PQ} = \mt{QP} = 0$ of a projection operator, which follows immediately from the definition of $\mt{P}$.

After we know the time evolution of an operator and the projection operator the correlation function is defined as
%
\begin{align}
	\mathcal{C}_{ij}(t) = \obraket{\mt{A}_{i}(t)}{\mt{A}_{j}(0)} \overset{\eqref{eq:scalar product Liouville space2.0}}{=} \frac{1}{\beta} \int\limits_{0}^{\beta} \dd{\lambda} \expval{\mt{A}_{i}^{\dag}(t) \mt{A}_{j}(i\lambda\hbar)},
	\label{eq:correlation function Liouville space}
\end{align}
%
where in the last step the definition of the scalar product is inserted.
Comparing equation \eqref{eq:correlation function Liouville space} with \eqref{eq:Kubo relaxation function 2.0} our choice of the correlation function is more clear.
The defined correlation function is proportional to the Kubo relaxation function.
In the memory-matrix-formalism we want to describe the reaction of an operator as a consequence of a switched off pertubation.
This is exactly desribes by the Kubo relaxation function, introduced in section \ref{sec:kubo relaxation function}
For $t=0$ the correlation function is also proportional to the Fourier transformated suscebtibility
%
\begin{align}
	\mathcal{C}_{ij}(t = 0) = \frac{1}{\beta} \Phi_{ij}(t = 0) = \frac{1}{\beta} \chi_{ij}(\omega = 0).
	\label{eq:relation between C, Phi and chi}
\end{align}
%
Equation \eqref{eq:formal solution of EM in L} is used to bring the time evolution of the correlation function in more suitable expression.
%
\begin{align}
	\mathcal{C}_{ij}(t) = \obraket{\mt{A}_{i}(0)}{\mt{A}_{j}(-t)} = \obra{\mt{A}_{i}(0)} e^{-it\mt{L}} \oket{\mt{A}_{j}(0)}
\end{align}
%
The obtained relation is transformed into frequency space using the Laplace transformation.
The used Laplace transformation is given by
%
\begin{align}
	f(\omega) = \int\limits_{0}^{\infty} \dd{t} e^{-i\omega t} f(t)
\end{align}
%
Equation for $\mathcal{C}_{ij}(t)$ is multiplied with $e^{i\omega t}$ and is intgrated from zero to infinty with resprct to $t$.
The Laplace transformed correlation function is given by
%
\begin{align}
	\mathcal{C}_{ij}(\omega) = \obra{\mt{A}_{i}} \int\limits_{0}^{\infty} \dd{t} e^{i(\omega-\mt{L})t} \oket{\mt{A}_{j}} = \obra{\mt{A}_{i}} \frac{i}{\omega - \mt{L}} \oket{\mt{A}_{j}}.
	\label{eq:correlation function frequency space}
\end{align}
%
For reasons of clarity the argument $t=0$ is not written at the basis vectors any more.
Now, the relation $\mt{L} = \mt{LQ} + \mt{LP}$ which follows immediatly using the definition of P and Q together with the identity $ (\mt{X} + \mt{Y})^{-1} = \mt{X}^{-1} - \mt{X}^{-1} \mt{Y} (\mt{X} + \mt{Y})^{-1}$ is used to simplify the correlation function.
We chose $\mt{X} = \omega - \mt{LQ}$ and $\mt{Y} = -\mt{LP}$.
%
\begin{align}
	\mathcal{C}_{ij}(\omega) &= \obra{\mt{A}_{i}} \frac{i}{\omega - \mt{LQ} - \mt{LP}} \oket{\mt{A}_{j}}
	\notag \\
	\Leftrightarrow\ \mathcal{C}_{ij}(\omega) &= \obra{\mt{A}_{i}} \frac{i}{\omega - \mt{LQ}} \oket{\mt{A}_{j}} + \obra{\mt{A}_{i}} \frac{1}{\omega - \mt{LQ}} \mt{LP} \frac{i}{\omega - \mt{L}} \oket{\mt{A}_{j}}
\end{align}
%
Both terms on the right hand side are considered seperatly.
The fraction of the first term is written as the geometric series.
It is assumed that $\frac{\mt{LQ}}{\omega} < 1$, which means that the pertubation is assumed to be small.
%
\begin{align}
	\frac{i}{\omega - \mt{LQ}} = \frac{i}{\omega} \bigg[1 + \frac{\mt{LQ}}{\omega} + \Big(\frac{\mt{LQ}}{\omega}\Big)^{2} + \dots \bigg]
\end{align}
%
Each term of the series in the squard brackets acting on the operator $\oket{\mt{A}_{j}}$.
This is the operator at time $t=0$, which means that no vertical component exists and therefore $\mt{Q}\oket{\mt{A}_{j}} = 0$.
Every term contains an operator $Q$, except of the first one.
The first term of the correlation function yields
%
\begin{align}
	\obra{\mt{A}_{i}} \frac{i}{\omega - \mt{LQ}} \oket{\mt{A}_{j}} = \frac{i}{\omega} \obraket{\mt{A}_{i}}{\mt{A}_{j}} = \frac{i}{\omega} \mathcal{C}_{ij}(0).
\end{align}
%
At the second term only the back is considered.
Here the explicit expression of the projection operator \ref{eq:projection operator2.0} is inserted.
This yields the definition of the Laplace transformed correlation function.
%
\begin{align}
	\mt{P} \frac{i}{\omega - \mt{L}} \oket{\mt{A}_{j}} = \sum\limits_{k,l} \oket{\mt{C}_{k}} \obraket{\mt{C}_{k}}{\mt{C}_{l}}^{-1} \obra{\mt{C}_{l}} \frac{i}{\omega - \mt{L}} \oket{\mt{A}_{j}} = \sum\limits_{k,l} \oket{\mt{C}_{k}} \mathcal{C}_{kl}^{-1}(0) \mathcal{C}_{lj}(\omega)
\end{align}
%
Both simplifications are inserted back and the correlation function is get the formal expression:
%
\begin{align}
	\mathcal{C}_{ij}(\omega) = \frac{i}{\omega} \mathcal{C}_{ij}(0) + \sum\limits_{k,l} \obra{\mt{A}_{i}} \frac{1}{\omega - \mt{LQ}} \mt{L} \oket{\mt{C}_{k}} \mathcal{C}_{kl}^{-1}(0) \mathcal{C}_{lj}(\omega).
\end{align}
%
In a last step, the fraction in the second term is more simplified.
Therefore, our expression is multiplied with $\omega$ and the null $\mt{LQ} - \mt{LQ}$ is added in the nominator at the fraction.
The rearrangement of the fractions yields the following algebraic matrix equation for the correlation function.
%
\begin{align}
	&\omega \mathcal{C}_{ij}(\omega) = i \mathcal{C}_{ij}(0) + \sum\limits_{k,l} \obra{\mt{A}_{i}} \frac{\omega}{\omega - \mt{LQ}} \mt{L} \oket{\mt{C}_{k}} \mathcal{C}_{kl}^{-1}(0) \mathcal{C}_{lj}(\omega)
	\notag \\
	\Leftrightarrow\ &\omega \mathcal{C}_{ij}(\omega) = i \mathcal{C}_{ij}(0) + \sum\limits_{k,l} \obra{\mt{A}_{i}} 1 + \frac{\mt{LQ}}{\omega - \mt{LQ}} \mt{L} \oket{\mt{C}_{k}} \mathcal{C}_{kl}^{-1}(0) \mathcal{C}_{lj}(\omega)
	\notag \\
	\Leftrightarrow\ &\omega \sum\limits_{l} \delta_{il} \mathcal{C}_{lj}(\omega) = i \frac{1}{\beta} \chi_{ij}(0) + \sum\limits_{l} \Big[\Omega_{il} -i \Sigma_{il}(\omega)\Big]  \mathcal{C}_{lj}(\omega)
	\notag \\
	\Leftrightarrow\ &\sum\limits_{l} \Big[\omega \delta_{il} - \Omega_{il} + i \Sigma_{il}(\omega)\Big] \mathcal{C}_{lj}(\omega) = i \frac{1}{\beta} \chi_{ij}(0)
	\label{eq:algebraic equation for C}
\end{align}
%
where equation \eqref{eq:relation 2 between Phi and chi} is used and the abbreviations
%
\begin{align}
	\Omega_{il} := \beta \sum\limits_{k} \obra{\mt{A}_{i}} \mt{L} \oket{\mt{C}_{k}} \chi_{kl}^{-1}(0)
	\qq{and}
	\Sigma_{il}(\omega) := i \beta \sum\limits_{k} \obra{\mt{A}_{i}} \frac{\mt{LQ}}{\omega - \mt{LQ}} \mt{L} \oket{\mt{C}_{k}} \chi_{kl}^{-1}(0)
\end{align}
%
are defined.
Equation \eqref{eq:algebraic equation for C} is mostly the final form of our correlation function.
This expression is an exact formula to determine the correlation function between two variables in a system.
Only in the explicite computation, using diagrammatic pertubation theory for example, assumptions and approximations are made.
The sums over $k$ and $l$ are originated from the utilization of the projection operator.
Therefore, each sum is summarized over all operators included in the set $\{\mt{C}_{i}\}$ of selected operators.
The indicies $i$ and $j$ our chosen out of this set as well.
Equation \eqref{eq:algebraic equation for C} yields therefore a $n^{2}$ algebraic equations, if $n$ is the number of operators in $\{\mt{C}_{i}\}$.

It is useful to write the even defined abbreviations in another form for our later computations.
For $\Omega_{il}$, we use equation \eqref{eq:Heisenberg equation of motion in Liouville space} to write the time derivative of an operator instead of the Liouville operator.
%
\begin{align}
	\Omega_{il} = i \beta \sum\limits_{k} \obraket{\dot{\mt{A}}_{i}}{\mt{C}_{k}} \chi_{kl}^{-1}(0).
	\label{eq:Omega}
\end{align}
%
For the rearrangement of the second abbreviation, equation \eqref{eq:Heisenberg equation of motion in Liouville space} is used as well.
Furthermore, the fraction is written as the geometric series.
In every term the relation $\mt{Q} = \mt{Q}^{2}$ is inserted.
After factorizing one $\mt{Q}$ to each vector operator the geometric series is written back as a fraction.
%
\begin{align}
	&\Sigma_{il}(\omega) = \frac{i \beta}{\omega} \sum\limits_{k} \obra{\dot{\mt{A}}_{i}} \mt{Q} \Big[1 + \frac{\mt{LQ}}{\omega} + \Big(\frac{\mt{LQ}}{\omega}\Big)^{2} + \dots\Big] \oket{\dot{\mt{C}}_{k}} \chi_{kl}^{-1}(0)
	\notag \\
	\Leftrightarrow\ &\Sigma_{il}(\omega) = \frac{i \beta}{\omega} \sum\limits_{k} \obra{\dot{\mt{A}}_{i}} \mt{Q}^{2} + \frac{\mt{Q^{2}LQ^{2}}}{\omega} + \frac{\mt{Q^{2}LQ^{2}LQ^{2}}}{\omega^{2}} + \dots\oket{\dot{\mt{C}}_{k}} \chi_{kl}^{-1}(0)
	\notag \\
	\Leftrightarrow\ &\Sigma_{il}(\omega) = \frac{i \beta}{\omega} \sum\limits_{k} \obra{\dot{\mt{A}}_{i}} \mt{Q} \Big[1 + \frac{\mt{QLQ}}{\omega} + \Big(\frac{\mt{QLQ}}{\omega}\Big)^{2} + \dots\Big] \mt{Q} \oket{\dot{\mt{C}}_{k}} \chi_{kl}^{-1}(0)
	\notag \\
	\Leftrightarrow\ &\Sigma_{il}(\omega) = i \beta \sum\limits_{k} \obra{\dot{\mt{A}}_{i}} \mt{Q} \frac{1}{\omega - \mt{QLQ}} \mt{Q} \oket{\dot{\mt{C}}_{k}} \chi_{kl}^{-1}(0)
	\label{eq:Sigma(z)}
\end{align} 
%
After all this exhausting mathematical and algebraical conversions the correlation function in the memory matrix formalsim is in a useful and workable form.
In equation \eqref{eq:algebraic equation for C} the abbreviations is combined to one function $\mt{M}_{il}(\omega) := \Sigma_{il}(\omega) +i \Omega_{il}$.
The symbol $\Sigma$ is selected in dependence on the quantum mechanical self energy.
The function $\mt{M}(\omega)$ is called the mass operator in quantum field theory and the memory function in non-equilibrium physics.

Let us discuss the physical meaning of $\Omega$ and $\Sigma(\omega)$ in more detail.
The quantity $\Omega$ always vanishs in the case, if the considered Hamiltonian possesses time reversal symmetry and if the operators $\mt{A}_{i}$ and $\mt{A}_{k}$ transform with the same sign under time reversal symmetry.
Under these conditions $\obraket{\dot{\mt{A}}_{i}}{\mt{A}_{k}} = 0$.
This assertion is immediatly proven extensivly in the section below.
In this case, the memory function is solely given by the function $\Sigma(\omega)$.
If we compare equation \eqref{eq:Sigma(z)} with the definition of the correlation function \eqref{eq:correlation function Liouville space} the string analogy is visable.
$\Sigma(\omega)$ is different in two aspects.
$\mt{Q} \oket{\dot{\mt{A}}}$ forms the basis vectors of the expectation value, which is perpendicular to $\oket{\mt{A}}$.
On the other hand only the reduced Liouville operator $\mt{QLQ}$ contribute to the expectation value.

The latter one projects at the part of the full Liouville operator $\mt{L}$, which causes the intrinsic fluctuations of the operator $\mt{A}$.
This means that the function $\Sigma(\omega)$ describes the dynamic of the operators.
In other words the operators $\mt{QLQ}$ describes the internal dynamics of all other degrees of freedom of the system excluded $\mt{A}$.
This is called the \emph{bath}.
The coupling to the bath is characterised by $\mt{Q} \oket{\dot{\mt{A}}}$ and is clearly changed the dynamics of $\mt{A}$.
%
%
\subsection{Time Reversal Symmetry}
\label{subsec: time reversal symmetry}
%
%
Even above the assertion is postulated that the quantity $\Omega_{il}$ vanishs, if the considered Hamiltonian is symmetric and if the operators $\dot{\mt{A}}$ and $\mt{A}$ have different sign under time reversal symmetry.
In the following section the evidence of this statement is proven.
Our starting point is the introduction of the time reversal operator $\mt{T}$ by the transformation rule
%
\begin{align}
	\mt{A}(t) \to \mt{A}'(t) = \mt{T} \mt{A}(t) \mt{T}^{-1} = \epsilon_{\mt{A}} \mt{A}(-t),
\end{align}
%
where $\epsilon_{A}$ supposees two different values, $+1$ or $-1$.
The first one is taken by the physical quantity as position or electrical field, while the latter is taken by physical quantity as momentum, angular momentum or magnetic field.
The action of $\mt{T}$ with respect to the time evolution of an operator is investigated firsly.
%
\begin{align}
	\mt{T} e^{i\mt{H}t/\hbar} \mt{T}^{-1} = e^{-i\mt{H}t/\hbar}
\end{align}
%
The Hamiltonian is assumed to be invariant under time reversal symmetry.
The only changed quantity is therefore the explicit time argument $t$.
The action of the time reversal operator on the time derivative of the time evolution of an operator is given by
%
\begin{align}
	\mt{T} \pdv{t} e^{i\mt{H}t/\hbar} \mt{T}^{-1} = \frac{i}{\hbar} \mt{TH} e^{i\mt{H}t/\hbar} \mt{T}^{-1} = \frac{i}{\hbar} \mt{THT}^{-1} \mt{T} e^{i\mt{H}t/\hbar} \mt{T}^{-1} = \frac{i}{\hbar} \mt{H} e^{-i\mt{H}t/\hbar}.
\end{align}
%
In the second step, the unit elememt $\mathds{1} = \mt{TT}^{-1}$ is inserted.
To get the commutator relation between $\mt{T}$ and $\mt{H}$, the time variable $t$ is set to zero and the equation is multiplied by $\mt{T}$ from the right.
%
\begin{align}	
	\comm{\mt{H}}{\mt{T}} = 0.
\end{align}
%
This is aquivalent to the assumption of an invariant Hamiltonian with respect to time reversal symmetry.
The expectation value of a Hermitain operator is manipulated with the time reversal operator $\mt{T}$.
%
\begin{align}
	\expval{\mt{B}} = \frac{1}{Z} \Tr{e^{-\beta \mt{H}} \mt{T} \mt{B} \mt{T}^{-1}} = \expval{\big(\mt{TBT}^{-1}\big)^{\dag}}
	\label{eq:time reversal expectation value}
\end{align}
%
Here the invariance of the trace with respect to cycling permutation and the commutator relation between $\mt{T}$ and $\mt{H}$ is used.
The anti-unitarity of the time reversal operator and the hermiticity of $\mt{B}$ is further utilized.
The same is done with the commutator between two Hermitian operators.
%
\begin{align}
	\bigg(\mt{T} \comm{\mt{A}(t)}{\mt{B}(t')} \mt{T}^{-1}\bigg)^{\dag} &= \epsilon_{\mt{A}} \epsilon_{\mt{B}} \bigg(\comm{\mt{A}(-t)}{\mt{B}(-t')}\bigg)^{\dag} = - \epsilon_{\mt{A}} \epsilon_{\mt{B}} \comm{\mt{A}(-t)}{\mt{B}(-t')}
	\label{eq:time reversal commutator}
\end{align}
%
As seen in equation \eqref{eq:Omega}, $\Omega_{il}$ is proportional to the correlation function $\obraket{\dot{\mt{A}}_{i}}{\mt{A}_{k}}$ between a time derivative quantity $\dot{\mt{A}}_{i}$ and the quantity $\mt{A}_{k}$.
The use of equation \eqref{eq:splitting susceptibility into real and imaginary part} yields
%
\begin{align}
	i \beta \obraket{\dot{\mt{A}}_{i}}{\mt{A}_{k}} = i \chi_{\dot{\mt{A}}_{i} \mt{A}_{k}}(\omega=0) = i\ \PV{\int\limits_{-\infty}^{\infty}} \frac{\dd{\omega'}}{\pi} \frac{\chi''_{\dot{\mt{A}}_{i} \mt{A}_{k}}(\omega')}{\omega'} - \lim\limits_{\omega \to 0} \chi''_{\dot{\mt{A}}_{i} \mt{A}_{k}}(\omega).
	\label{eq:equation for Omega}
\end{align}
%
Here the dissipative susceptibility $\chi''_{\dot{\mt{A}}_{i} \mt{A}_{k}}(\omega)$ occurs.
The difference to $\chi''_{\mt{A}_{i} \mt{A}_{k}}(\omega)$ is the time derivative of an operator.
Nevertheless, the one dissipative susceptibilities is achieved from the other one.
To find the relation between both the derivative of \eqref{eq:dissipative susceptibility} is evaluated.
%
\begin{align}
	\dv{t} \chi''_{\mt{A}_{i} \mt{A}_{k}}(t) = \frac{1}{2} \expval{\comm{\dot{\mt{A}}_{i}(t)}{\mt{A}_{k}(0)}} = \chi''_{\dot{\mt{A}}_{i} \mt{A}_{k}}(t)
\end{align}
%
We express both susceptibilities by their Fourier transformions and we obtain
%
\begin{align}
	\chi''_{\dot{\mt{A}}_{i} \mt{A}_{k}}(\omega) = - i \omega \chi''_{\mt{A}_{i} \mt{A}_{k}}(\omega)
\end{align}
%
This is inserted into \eqref{eq:equation for Omega}, which yields
%
\begin{align}
	i \beta \obraket{\dot{\mt{A}}_{i}}{\mt{A}_{k}} = \PV{\int\limits_{-\infty}^{\infty}} \frac{\dd{\omega'}}{\pi} \chi''_{\mt{A}_{i} \mt{A}_{k}}(\omega'),
	\label{eq:equation for Omega depend on Chi''}
\end{align}
%
where the limit $\omega$ is evaluated.
This result entails two very important advantages.
The physical meaning of the quantity $\Omega_{il}$ becomes clearer, since $\Omega_{il}$ is associated with dissipative processes by $\chi''_{\mt{A}_{i} \mt{A}_{k}}(\omega')$.
On the other hand the founded expression establishs the possibility to analyse the behaviour of $\Omega_{il}$ under time reversal symmetry.
The dissipative susceptibility in equation \eqref{eq:dissipative susceptibility} is now observed.
The expectation value is rewriten using equation \eqref{eq:time reversal expectation value} and \eqref{eq:time reversal commutator}.
%
\begin{align}
	\chi''_{\mt{A}_{i} \mt{A}_{k}}(t-t') = \frac{1}{2} \expval{\comm{\mt{A}_{i}(t)}{\mt{A}_{k}(t')}}= -\epsilon_{\mt{A}_{i}} \epsilon_{\mt{A}_{k}} \chi''_{\mt{A}_{i} \mt{A}_{k}}(t'-t),
\end{align}
%
where the relation \eqref{eq:complex conjugated of dissipative susceptibility} is used.
The Laplace transformation of this equation yields
%
\begin{align}
	\chi''_{\mt{A}_{i} \mt{A}_{k}}(\omega) = -\epsilon_{\mt{A}_{i}} \epsilon_{\mt{A}_{k}} \chi''_{\mt{A}_{i} \mt{A}_{k}}(-\omega) = \epsilon_{\mt{A}_{i}} \epsilon_{\mt{A}_{k}} \chi''_{\mt{A}_{i} \mt{A}_{k}}(\omega),
\end{align}
%
where the antisymmetry of the commutator with respect of interchanging both operators is utilized.
Two cases has to be investigated analyzing the analytical properties of $\chi''_{\mt{A}_{i} \mt{A}_{k}}(\omega)$, which are $\epsilon_{\mt{A}_{i}} = \epsilon_{\mt{A}_{k}}$ and $\epsilon_{\mt{A}_{i}} \neq \epsilon_{\mt{A}_{k}}$.
The analysis gives the required properties to compute the integral over the dissipative susceptibility.
%
\paragraph{1. case:} $\epsilon_{\mt{A}_{i}} = \epsilon_{\mt{A}_{k}}$\\
%
This yields $\chi''_{\mt{A}_{i} \mt{A}_{k}}(\omega) = \chi''_{\mt{A}_{k} \mt{A}_{i}}(\omega)$, which means that the dissipative susceptibility is symmetrical under interchanging $\mt{A}_{i}$ and $\mt{A}_{k}$.
The dissipative sysceptibility is further an antisymmetrical function with respect to $\omega$, since $\chi''_{\mt{A}_{i} \mt{A}_{k}}(\omega) = -\chi''_{\mt{A}_{i} \mt{A}_{k}}(-\omega)$.
The complex conjugated of $\chi''_{\mt{A}_{i} \mt{A}_{k}}(\omega)$ yields, that the dynamical susceptibility is a real number.
%
\begin{align}
	\Big(\chi''_{\mt{A}_{i} \mt{A}_{k}}(\omega)\Big)^{*} = -\int\limits_{-\infty}^{\infty} \dd{t} e^{-i\omega(t-t')} \chi''_{\mt{A}_{i} \mt{A}_{k}}(t-t') = -\chi_{\mt{A}_{i} \mt{A}_{k}}(-\omega) = \chi_{\mt{A}_{i} \mt{A}_{k}}(\omega)
\end{align}
%
Here equation \eqref{eq:complex conjugated of dissipative susceptibility} is used.
%
\paragraph{2. case:} $\epsilon_{\mt{A}_{i}} \neq \epsilon_{\mt{A}_{k}}$\\
%
If the sign of $\mt{A}_{i}$ and $\mt{A}_{k}$ is different under time reversal symmetry, the dissipative susceptibility is antisymmetric under the interchange of both operators.
This yields $\chi''_{\mt{A}_{i} \mt{A}_{k}}(\omega) = -\chi''_{\mt{A}_{k} \mt{A}_{i}}(\omega)$.
For the same reason $\chi''_{\mt{A}_{i} \mt{A}_{k}}(\omega)$ is a symmetrical function with respect to $\omega$, since $\chi''_{\mt{A}_{i} \mt{A}_{k}}(\omega) = \chi''_{\mt{A}_{i} \mt{A}_{k}}(-\omega)$.
Towards the first case, the dissipative susceptibility is an imaginary number, ensuring by the complex conjugation of $\chi''_{\mt{A}_{i} \mt{A}_{k}}(\omega)$.
%
\begin{align}
	\Big(\chi''_{\mt{A}_{i} \mt{A}_{k}}(\omega)\Big)^{*} = -\int\limits_{-\infty}^{\infty} \dd{t} e^{-i\omega(t-t')} \chi''_{\mt{A}_{i} \mt{A}_{k}}(t-t') = -\chi_{\mt{A}_{i} \mt{A}_{k}}(-\omega) = -\chi_{\mt{A}_{i} \mt{A}_{k}}(\omega)
\end{align}
%
The integral in equation \eqref{eq:equation for Omega depend on Chi''} is now solvble for one of these cases.
We see that the integral vanishs in the first case, since the suscebtibility is an odd function.
This means that $i \beta \obraket{\dot{\mt{A}}_{i}}{\mt{A}_{k}}$ is always zero, if the operators $\mt{A}_{i}$ and $\mt{A}_{k}$ have the same signature with respect to time reversal symmetry and the Hamiltonian is invariant under time reversal symmetry.

%
%
\section{Formula of the Static Conductivity}
\label{sec:formula static conductivity}
%
%
In this section, the formula for the static electrical conductivity is derivated.
The considered system is the spin-fermion-model, described by the Hamiltonain H, pertubated by umklapp scattering $\mt{H}_{\mt{umklapp}}$.
Both Hamiltonias are denoted in \ref{ch:spin fermion model}.
The static electrical conductivity is given by the current current correlation function, as shown in \eqref{eq:general static condictivity}.
To get the correlation function in the memory-matrix-formalism, equation \eqref{eq:algebraic equation correlation function} has to be solved.

Equally to the discussion in chapter \ref{ch:infinite conductivity}, momentum P and current J are chosen as the subspace operators of the projection operator.
Again, J and P possesses the same sign and the Hamiltian is still invariant unter time reversal symmetry.
as a consequnce the quantity $\Omega_{il}$ in equation \eqref{eq:algebraic equation correlation function} vanishes.
In comparsion to the discussion in chapter \ref{ch:infinite conductivity}, momentum is now unconserved.
The time derivatives with respect to P in $\Sigma_{il}(\omega)$ do not vanishes.
The condition that $\mt{Q}\toket{\dot{J}}$ vanishes is still valid as discussed in chapter \ref{ch:infinite conductivity}.
Since the time derivative of P does not vanish, the object $\mt{Q}\toket{\dot{P}}$ occurs.
The J-P subspace is defined with respect to the unpertubated Hamiltonian H, where momentum is conserved.
The value of the time derivative of P is only determined by the pertubation and lies completly out of the J-P subspace
Therefore, $\mt{Q}\toket{\dot{P}} = \toket{\dot{P}}$ is valid.
For the same reason, the reduced Liouville operator QLQ is assumed to be as L.
All other discussed conditions in chapter \ref{ch:infinite conductivity} are still valid.
The algebraic matrix equation for the correlation function is given by
%
\begin{align}
	\begin{pmatrix}
	\omega & 0 \\
	-i\Sigma_{\mt{PJ}}(\omega) & \omega - i\Sigma_{\mt{PP}}(\omega)
	\end{pmatrix}
	\cdot
	\begin{pmatrix}
	\mathcal{C}_{\mt{JJ}}(\omega) &  \mathcal{C}_{\mt{JP}}(\omega) \\
	\mathcal{C}_{\mt{PJ}}(\omega) &  \mathcal{C}_{\mt{PP}}(\omega)
	\end{pmatrix}
	=
	\frac{i}{\beta}
	\begin{pmatrix}
	\chi_{\mt{JJ}}(0) &  \chi_{\mt{JP}}(0) \\
	\chi_{\mt{PJ}}(0) &  \chi_{\mt{PP}}(0)
	\end{pmatrix}
	\label{eq:matric equation correlation function unconserved momentum}
\end{align}
%
The current current correlation function is needed for computing the electrical conductivity.
In the memory-matrix-formalism, the definition of this correlation function is given by the following formula in frequancy space.
%
\begin{align}
	\mathcal{C}_{\mt{JJ}}(\omega) = \obra{\mt{J}} \frac{i}{\omega - \mt{L}} \oket{\mt{J}}
\end{align}
%
Equally to the procedure in chapter \ref{ch:infinite conductivity} the current operator is expressed as an parallel and perpendicular component.
The parallel component is still identified with the projection from $\toket{J}$ onto $\toket{P}$.
The current operator is seperated in the expression above, producing for new correlation functions.
Two of them are zero, since $\toket{\mt{J}_{\mid\mid}}$ and $\toket{\mt{J}_{\bot}}$ are orthogonal.
As discussed in chapter \ref{ch:infinite conductivity}, the correlation function containing the perpendicular part represents the noisy background.
In the further calculation this term is dropped, since the origin is chosen at this value.
A detailed discussion is given in \cite{Jung}.
The current curent correlation function is therefore given by
%
\begin{align}
	\mathcal{C}_{\mt{JJ}}(\omega) = \obra{\mt{J}_{\mid\mid}} \frac{i}{\omega - \mt{L}} \oket{\mt{J}_{\mid\mid}} = \frac{\vert\chi_{\mt{PJ}}\vert^{2}}{\vert\chi_{\mt{PP}}\vert^{2}} \mathcal{C}_{\mt{PP}}(\omega)
	\label{eq:JJ correlation function}
\end{align}
%
Equation \eqref{eq:parallel current as projection} is used in the second step.
The momentum momentum correlation function is readed out of equation \eqref{eq:matric equation correlation function unconserved momentum}.
Therefore the inverse of the memory matrix is multiplied from the left.
The following expression is obtained for the momentum momentum correlation function
%
\begin{align}
	\mathcal{C}_{\mt{PP}}(\omega) = \frac{i}{\beta} \cdot \frac{i \Sigma_{\mt{PJ}}(\omega)  \chi_{\mt{JP}}(0)}{\omega\big(\omega - i\Sigma_{\mt{PP}}(\omega)\big)} + \frac{i}{\beta} \cdot \frac{\chi_{\mt{PP}}(0)}{\omega - i\Sigma_{\mt{PP}}(\omega)} \approx \frac{i}{\beta} \cdot \frac{i \chi_{\mt{PP}}(0)}{\Sigma_{\mt{PP}}(\omega)}
\end{align}
%
In the last step, the limit of small frequencies $\omega$ is made.
The first term is proportional to $\omega^{-2}$ and is therefore assmued to be small, comparing to the second one.
This equation is inserted in \eqref{eq:JJ correlation function}, while this one is then again inserted in \eqref{eq:general static condictivity}.
This yields the following formula for the static electrical conductivity:
%
\begin{align}
	\sigma_{\mt{dc}} = \lim\limits_{\omega \to 0} \beta \mathcal{C}_{\mt{JJ}}(\omega) = \frac{i}{\beta} \lim\limits_{\omega \to 0} \vert\chi_{\mt{PJ}}\vert^{2} \obra{\dot{\mt{P}}} \frac{1}{\omega - \mt{L_{0}}} \oket{\dot{\mt{P}}}^{-1} = -\beta^{-1} \lim\limits_{\omega \to 0} \vert\chi_{\mt{PJ}}\vert^{2} \mathcal{C}_{\dot{\mt{P}} \dot{\mt{P}}}^{-1}(\omega).
\end{align}
%
Here the definition \eqref{eq:Sigma(z)} of the memory function $\Sigma_{\mt{PP}}(\omega)$ and the definition of the correlation function in fequency space \eqref{eq:correlation function frequency space} is used.
Above, it is discussed that the replacing of QLQ by L is valid.
The $\dot{\mt{P}}$-$\dot{\mt{P}}$ correlation function is now transformed in an integral equation.
In this representation the use of diagrammatic pertubation theory is possible.
%
\begin{align}
	\mathcal{C}_{\dot{\mt{P}} \dot{\mt{P}}}(\omega) &= \int\limits_{0}^{\infty} \dd{t} e^{i\omega t} \mathcal{C}_{\dot{\mt{P}} \dot{\mt{P}}}(t) = \beta^{-1} \int\limits_{0}^{\infty} \dd{t} e^{i\omega t} \int\limits_{0}^{\beta} \dd{\lambda} \expval{\dot{\mt{P}}^{\dag}(t) \dot{\mt{P}}(0)}
	\notag \\
	\Leftrightarrow\ \mathcal{C}_{\dot{\mt{P}} \dot{\mt{P}}}(\omega) &= \beta^{-1} \Phi_{\dot{\mt{P}} \dot{\mt{P}}}(\omega) = \frac{i\omega^{-1}}{\beta} \bigg[\chi_{\dot{\mt{P}} \dot{\mt{P}}}(\omega) - \underbrace{\chi_{\dot{\mt{P}} \dot{\mt{P}}}(\omega=0)}_{=0}\bigg]
	\notag \\
	\Leftrightarrow\ \mathcal{C}_{\dot{\mt{P}} \dot{\mt{P}}}(\omega) &= -\frac{\omega^{-1}}{\beta} \int\limits_{0}^{\infty} \dd{t} e^{i\omega t} \expval{\comm{\dot{\mt{P}}(t)}{\dot{\mt{P}}(0)}}_{0}
\end{align}
%
The correlation function is transformed into time space and the definition of the scalar product \eqref{eq:scalar product Liouville space2.0} is used.
This expression is equivalent to the Fourier transformed Kubo relaxation function.
In frequency space, the Kubo relaxation function is connected with the susceptibility by relation \eqref{eq:relation 2 between Phi and chi}.
Since the time derivative of P is zero at $t=0$, the static susceptibility vanishes.
The susceptibility is again transformed into time space and the definition \eqref{eq:dynamical susceptibilty} is used.

Inserting this expression in our obtained equation for the static electrical conductivity, the final form of their is given by
%
\begin{align}
	\sigma_{\mt{dc}} = \lim\limits_{\omega \to 0} \frac{\omega \vert \chi_{\mt{JP}}(\omega = 0) \vert^{2}}{\int\limits_{0}^{\infty} \dd{t} e^{i\omega t} \expval{\comm{\dot{\mt{P}}(t)}{\dot{\mt{P}}(0)}}_{0}},
	\label{eq:formula static conductivity}
\end{align}
%


































