%
%
%
\chapter{Commutator relation}
\label{appch:commutator}
%
%
%
The time derivative of the momentum and current operator are designated for the case of non-pertubation and pertubation in chapter \ref{ch:spin fermion model}.
In this appendix, the explicit computation is demonstrated of them.
Our starting point is the Lagrange density $\mathcal{L} = \mathcal{L}_{\Psi} + \mathcal{L}_{\Phi} + \mathcal{L}_{\Psi\Phi}$, as suggested in \cite{Patel&Sachdev}.
In Matsubara time $\tau = it$, the single Lagrange densities are given by
%
\begin{align}
	\mathcal{L}_{\Psi} = 
		\vb{\Psi}^{\dag}(\vb{x},\tau) 
		\twoTwoMatrix{\sigma_{0} (\partial_{\tau}-\mu_{0}+\xi_{a})}{0}{0}{\sigma_{0} (\partial_{\tau}-\mu_{0}+\xi_{b})}
		\vb{\Psi}(\vb{x},\tau)
\end{align}
%
\begin{align}
	\mathcal{L}_{\Phi} &= 
		\frac{1}{2} \big(\partial_{i}\Phi_{\mu}(\vb{x},\tau)\big) \big(\partial_{i}\Phi_{\mu}(\vb{x},\tau)\big) 
		+ 
		\frac{\epsilon}{2} \big(\partial_{\tau}\Phi_{\mu}(\vb{x},\tau)\big) \big(\partial_{\tau}\Phi_{\mu}(\vb{x},\tau)\big) 
		\notag \\&+
		\frac{u}{6} \big(\Phi_{\mu}(\vb{x},\tau) \Phi_{\mu}(\vb{x},\tau) + \frac{3}{g}\big)^{2}
\end{align}
\begin{align}
	\mathcal{L}_{\Psi\Phi} =
		\lambda\Phi_{\mu}(\vb{x},\tau) \Big[\Psi_{\mt{a}}^{\dag}(\vb{x},\tau) \sigma_{\mu} \Psi_{\mt{b}}(\vb{x},\tau) + \Psi_{\mt{b}}^{\dag}(\vb{x},\tau) \sigma_{\mu} \Psi_{\mt{a}}(\vb{x},\tau)\Big]
\end{align}
Here $\vb{\Psi} = (\Psi_{\mt{a}}, \Psi_{\mt{b}})$ and $\Psi_{\mt{a}}$, $\Psi_{\mt{b}}$ are two-component spinors and $\Phi_{\mu}$ is a SO(3) vector boson order parameter.
The chemical potential is denoted with $\mu_0$ and the Pauli matrix are labeled with $\sigma_{0}$ and $\sigma_{\mu}$.
$\epsilon$ is the squared inverse spin velocity $v_{\mt{S}}^{-2}$.
In the Lagrangian $\mathcal{L}_{\Phi}$, the last term is neglectable in the low-energy theory of the spin fermion model and is therefore dropped in the following.
The anisotropic parabolical dispersion relations $\xi_{\mt{a}}$ and $\xi_{\mt{b}}$ are given by 
%
\begin{align}
	\xi_{\mt{a}} = - \frac{\partial_{x}^{2}}{2m_{1}} - \frac{\partial_{y}^{2}}{2m_{2}} \qquad \xi_{\mt{b}} = - \frac{\partial_{x}^{2}}{2m_{2}} - \frac{\partial_{y}^{2}}{2m_{1}}
\end{align}
%
The Lagrangian for umklapp scattering is assumed to be 
%
\begin{align}
	\mathcal{L}_{\mt{umklapp}}= -\mt{J}(\vb{R}) \Phi_{\mu}(\vb{x},\tau) \Phi_{\mu}(\vb{x},\tau)
\end{align}
%
where $\mt{J}(\vb{R})$ is a coupling parameter and $\vb{R}$ is a lattice vector.
The following treatment is now to compute the Hamiltonian, momentum and current, using the energy-momentum-tensor
%
\begin{align}
	\mathcal{T}_{\mu\nu} = 
		\frac{1}{2} \sum\limits_{\{\zeta_{n}\}} \bigg[ 
		\Big(\frac{\partial\mathcal{L}}{\partial(\partial_{\mu}\zeta_{n})}\Big) (\partial_{\nu}\zeta_{n}) 
		+
		(\partial_{\nu}\zeta_{n})^{\dag} \Big(\frac{\partial\mathcal{L}}{\partial(\partial_{\mu}\zeta_{n})}\Big)^{\dag}
		\bigg]
\end{align}
%
and the formula for the current density
%
\begin{align}
	\mathcal{J}_{\mu} = -i \sum\limits_{\{\zeta_{n}\}} \epsilon(\zeta_{n}) \bigg[
		\Big(\frac{\partial\mathcal{L}}{\partial(\partial_{\mu}\zeta_{n})}\Big) \zeta_{n}
		-
		\zeta_{n}^{\dag} \Big(\frac{\partial\mathcal{L}}{\partial(\partial_{\mu}\zeta_{n})}\Big)^{\dag}
		\bigg],
\end{align}
%
where $\{\zeta_{n}\} = \{\Psi_{\mt{a}},\Psi_{\mt{b}}, \Phi_{\mu}\}$ and $\epsilon(\zeta_{n})$ is 0 in the case of real fields and 1 in the case of complex fields.
The energy and the $j$-component of the momentum is represented by $\mathcal{T}_{00}$ and $\mathcal{T}_{0j}$ of the energy-momentum-tensor, respectivily.
The $j$-component of the current is represented by $\mathcal{J}_{j}$.
In the spin-fermion-model, the bosonic field operator is considered as real, so that $\Phi_{\mu}^{\dag} = \Phi_{\mu}$.












































